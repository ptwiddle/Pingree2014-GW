\documentclass{amsart}


\linespread{1.2}

\theoremstyle{definition}
\newtheorem{dummy}{}[section]

\newtheorem{theorem}[dummy]{Theorem}
\newtheorem{definition}[dummy]{Definition}
\newtheorem{example}[dummy]{Example}
\newtheorem{question}[dummy]{Question}
\newtheorem{conjecture}[dummy]{Conjecture}

\newcommand{\I}{\mathcal{I}}
\newcommand{\OO}{\mathcal{O}}
\newcommand{\Map}{\textrm{Map}}
\newcommand{\proj}{\mathbb{P}}
\newcommand{\J}{\mathcal{J}}
\newcommand{\sm}{\text{sm}}
\newcommand{\Mbar}{\overline{\mathcal{M}}}
\newcommand{\M}{\mathcal{M}}
\newcommand{\Q}{\mathbb{Q}}
\newcommand{\C}{\mathbb{C}}
\newcommand{\Age}{\text{Age}}
\newcommand{\Fix}{\text{Fix}}
\newcommand{\one}{1}
\newcommand{\st}{\text{st}}
\newcommand{\ttt}{\mathbb{t}}
\newcommand{\ev}{\text{ev}}
\newcommand{\vir}{\text{vir}}

\newcommand{\Aut}{Aut}
\title{Quasimaps and mirror theorems for GIT quotients}
\author{Ionut Ciocan-Fontanine}


\begin{document}
\maketitle

We begin with some of Givental's early work on mirror symmetry, from around twenty years ago; in future lectures, we discuss how this can be generalized in various ways.

\subsection{Notation}

Let $X$ be a smooth projective variety over $\C$, and $\beta\in NE(X)$ a class in the cone of effective curves.

\begin{definition}
The \emph{Novikov ring} $\Lambda_X$ is the completion of the group ring of $NE(X)$:

$$\Lambda_X=\hat{\Q[NE(X)]}=\{\sum_\beta a_\beta q^\beta|a_\beta\in\Q\}$$
\end{definition}

We will use $H^*(X)=H^{even}(X,\Q)$ so we don't have to worry about signs.

The intersection form on cohomology is $\langle, rangle$; 


We will fix $T_q=\{\one,\dots, T_s\}$ a homogenous basis of $H^*$; and let $\{T^1,\dots, T^s\}$ the dual basis.

We will use $t=\sum t_i T_i$ a general element in cohomology.

\subsubsection{GW invariants}

We have $\Mbar_{g,k}(X,\beta)$, with universal family $\pi:\mathcal{C}\to\Mbar_{g,k}(X,\beta)$.  And $u:\mathcal{C}\to X$ the universal map.

$$\langle \psi_1^{a_1}\gamma_1\dots \psi_k^{a_k}\gamma_k\rangle_{g,k,\beta}
=\int_{[\Mbar_{g,k}(X,\beta)]^{\vir}} \prod \psi_i^{a_i} \prod \ev_i^* \gamma_i$$



Genus $g$ potential:

$$\mathcal{F}_g(\ttt)=\sum_{k,\beta} \frac{q^\beta}{k!} \langle \ttt(\psi_1),\dots,\ttt(\psi_k)\rangle_{g,k,\beta}$$

Also

$$\langle\langle \psi_1^{a_1}\gamma_1\dots \psi_k^{a_k}\gamma_k\rangle\rangle_{g,k}=\sum_{\beta,m}\frac{q^\beta}{m!}\langle \psi_1^{a_1}\gamma_1\dots \psi_k^{a_k}\gamma_k, \ttt(\psi_{k+1}),\dots,\ttt(\psi_{k+m})\rangle_{g,k+m,\beta}$$

\subsubsection{Noncompact targets}
Let $p:X\to X_0$ be a projective map with $X_0$ affine.  Let $T=(\C^*)^r$ act on $X$ and $X_0$ so that $p$ is $T$-equivariant.

Assume that $X_0^T$ is proper, and (and so a finite set).  Then $X^T$ and $\Mbar_{g,k}(X,\beta)^T$ are proper.  The evaluation maps, and the composition $\Mbar_{g,k}(X,\beta)\to X_0$ of $p$ with the evaluation maps are proper.

We define $H^*(X)=H^{even}_{T, loc}(X,\Q)$ to be the localized even $T$-equivariant cohomology.

Let $\iota:X^T\to X$ be the inclusion of the fixed point locus, and $N$ the equivariant normal bundle of $X^T$ in $X$; in this case we have pairing

$$\langle, \rangle=\int_{X^T}\frac{\iota^*(\gamma\delta)}{e(N)}$$
taking values in the localized equivariant cohomology of a point.

Our equivariant Gromov-Witten invariants are

$$\langle \psi_1^{a_1}\gamma_1\dots \psi_k^{a_k}\gamma_k\rangle_{g,k,\beta}
=\int_{[\Mbar_{g,k}(X,\beta)]^{\vir}} \frac{\prod \psi_i^{a_i} \prod \ev_i^* \gamma_i}{e(N)}=$$
MISSED THIS LAST LINE.


\section{The J-function}

\begin{definition}
The \emph{$J$ function of $X$} is

$$\J(q,t,z)=\one+\frac{t}{z}+\sum_{\beta,k}\frac{q^\beta}{k!}\ev_{1*}\left(\frac{[\Mbar_{0,k}(X,\beta)]^\vir}{z(z-\psi)}\cap \prod_{i=1}^k \ev^*_{i+1}(t)\right)$$
$$=\one +\frac{t}{z}+\sum_i T_i\langle\langle \frac{T^i}{z(z-\psi)}\rangle\rangle_{0,1}(t)$$

where $\psi=\psi_1$.
\end{definition}


The first two terms should be understood as contributions from unstble moduli spaces.

Note that $\J\in H^*(X)[[\frac{1}{z}]][[1]][[t_i]]$.

Givental's small $\J$ is $\J(q,t_\sm,z)$, where $t_\sm=t|_{H^{\leq 2}}$ is restricted to point or divisor insertions.

By the string and divisor equations, we have
\begin{align*}
\J(q,t_\sm,z)&=
e^{t_\sm/z}\sum_\beta q^\beta e^{\int_\beta t_\sm} \ev_{1*} \frac{[\Mbar_{0,1}(X,\beta)]^\vir}{z(z-\psi)} \\
&=e^{t_\sm/z}\sum_\beta q^\beta e^{\int_\beta t_\sm} \sum_i T_i\left\langle \frac{T_i}{z(z-\psi)}\right\rangle_{0,1,\beta}
\end{align*}

The point of the $\J$ function is that it determines $\mathcal{F}_0$ by splitting and string equation and topological recursion relations.

Furthermore, if $H^*$ is generated as an algebra by $H^2$, then $\J_\sm$ determines $\J$ (Lee-Pandharipande, Bertram- reconstruction functions).  So we'd really like to have an understanding on of the $\J$ function.

\subsection{}
An early insight of Givental is that it is useful to consider the $\C^*$-equivariant theory of maps from pointed \emph{parametrized} $\proj^1$ to $X$.  Comactify by allowing unparametrized ``bubbles'' to get \emph{graph space}:

$$G_{k,\beta}(X)=\Mbar_{0,k}\left(X\times \proj^1,(\beta,1)\right)
=\left\{(C,\phi)|u:C\to X, \phi:C\to\proj^1\right\}$$
so there exists a component $C_0\subset C$ such that $\phi$ restricted to $C_0$ is an isomorphism to $\proj^1$, and stabilizes $C\setminus C_0$.

Then $\C^*$ acts on $\proj^1$ in the standard way, hence on $G_{k,\beta}(X)$.

The equivariant cohomology $H^*(\mathcal{B}\C^*)=\C[z]$.

The fixed point set will satisfy several properties:

\begin{enumerate}
\item  $C_0\cap \overline{(C\setminus C_0)}\subseteq\{0,\infty\}$
\item There will be no markings on $C_0\setminus \{0,\infty\}$
\item $u(C_0)=\text{pt}\in X$
\end{enumerate}


Let $C_1$ be the union of components that map to $0\in\proj^1$, and $C_2$ the union of components mapping to $\infty$.  Let $\beta_1,\beta_2$ be the corresponding degrees of maps of curves ot $X$, and $k_1$ and $k_2$ the number of marked points on each components; we have

\begin{align*}
\beta_1+\beta_2&=\beta \\
k_1+k_2&=k
\end{align*}

The points where $C_1$ and $C_2$ attach to $C_0$ give extra markings on these curves, which we will denote by $\bullet$.  

Thus, the fixed point loci correspond to pairs of maps $u_1:C_1\to X$ and $u_2:C_2\to X$, so that $u_1(\bullet)=u_2(\bullet)=u(C_0)$.

And so we have 

$$F^{k_1,\beta_1}_{k_2,\beta_2}=\Mbar_{0,\bullet+k_1}(X,\beta_1)\times_X \Mbar_{0,\bullet+k_2}(X,\beta_2)$$
with the following conventions for unstable spaces.
$$\Mbar_{0,\bullet+1}(X,0)=\Mbar_{0,\bullet+0}(X,0)=X, \ev_\bullet=\text{is}$$

We have

$$[F^{k_1,\beta_1}_{k_2,\beta_2}]^\vir=\Delta^!\left([\Mbar_1]^\vir\otimes[\Mbar_2]^\vir\right)$$
and the euler class of the furture normal bundle is

$e(N^\vir_{F^{k_1,\beta_1}_{k_2,\beta_2}})$ 
is the product of two contributions, depending on $k_1$ and $\beta_1$, where

NORMAL BUNDLE CONTRIBUTIONS


Let $F_0=F^{k,\beta}_{0,0}$.  

Then

$$\text{Res}_{F_0} \left([G_{k,\beta}(X)]^\vir\cap \prod \tilde{\ev}^*_i(t\otimes\one_{\proj^1})\right)
=\frac{[\Mbar_{0, \bullet+k} (X,\beta)]^\vir}{z(z-\psi)}\cap\prod \ev^*_i(t)$$


And hence:

$$\J(q,t,z)=\sum_{\beta,k} \frac{q^\beta}{k!}(\ev_\bullet)_*\text{Res}_{F_0}\left([G_{k,\beta}(X)]^\vir\cap \prod \tilde{\ev}^*_i(t\otimes\one_{\proj^1})\right)$$


It is general hard to calculate the big $\J$ function.  For small $\J$, (or $\J|_{t=0}$), only need $G_\beta(X)=G_{0,\beta}(X)$, which contains $\text{Map}_\beta(\proj^1,X)$.

Let us restrict to $X=\proj^{n-1}$.  Let $H\in H^2(\proj^{n-1})$ be the hyperplane class.

IN this case $t=t_1\one+t_2H+\cdots+t_nH^{n-1}$.

\begin{theorem}[Givental]
$$\J(q,0,z)=\sum_{d\geq 0} q^d \frac{1}{\prod_{j=1}^d(H+jz)^n}$$
\end{theorem}

\begin{proof} 
There is the following cheap proof.

Use $\J(q,t_\sm,z)$ and a modified right hand side, containing some exponentialfactors.

Check that the right hand side satisfies the quantum differential equation and has the correct initial conditions.
\end{proof}

This proof is not particularly illuminating; we now sketch another proof, that first appears in a paper of Bertram, but perhaps was known in some way earlier:

First, the right hand side doesn't come from a dream, but from the geometry of a \emph{different} compactification of $\text{Map}_d(\proj^1,\proj^{n-1})$.

From basic algebraic geometry, maps to project space are given by line bundles and sections, and in particular we have

$$\Map_d(\proj^1,\proj^{n-1})=\{(u_1,\dots,u_n)|u_j\in H^0(\OO_{\proj^1}(d))\text{ with no common zeroes}\}/\C^*$$

There is an obvious way to compactify this space; just drop the condition that the $u_i$ have no common zeros.

So, this sits inside:

$$\proj^{n(d+1)-1}=\proj(H^0(\OO_{\proj^1}(d)\otimes\C^n)$$
we define the as the ``quasimap graph space'' and denote it by $QG_{0,d}(\proj^{n-1})$.

Drinfeld coined the name ``quasimaps'' earlier, there is a question about how this is related to a linear sigma model (which isn't quite the same because it should include higher genus?)

So, let $[u]=[u_1,\dots,u_n]:\proj^1\to\proj^{n-1}$ a rational map, possibly with base points.

It extends to a regular map $[u_\text{reg}]:\proj^1\to\proj^{n-1}$ of degree $d_{\text{reg}}\leq d$, with 
$$d-d_{\text{reg}}=\sum_{p\in\proj^1}\text{order of the common zeroes of the $u_i$ at $p$}$$

Since $\C^*$ acts on $\proj^1$, we get an induced action on $QG_{0,d}(\proj^{n-1})$.  The fixed points are rational maps $[u]$ with base locus contained in $\{0,\infty\}$ and $[u_{\text{reg}}]$ constant (i.e. $d_{\text{reg}}=0$).

Since $d_1+d_2=d$, let $[u]\in \hat{F}^{d_1}_{d_2}$; then $[u]=[a_1x^{d_1}y^{d_2},\dots,a_nxd^{d_1}y^{d_2}]$, and so $\ev_\bullet$ gives an isomorphism to $\proj^{n-1}$, that takes $[u]$ to $[a_1,\dots, a_n]$.  

In particular, $\hat{F}_0=\hat{F}^d_0$.

There is an ``obvious'' $\C^*$-equivariant birational contraction:

$$\phi:G_{0,d}(\proj^{n-1})\to QG_{0,d}(\proj^{n-1})$$

$\phi$ contracts all the components other than $C_0$.

DRAWING AND FURTHER DESCRIPTION

Thus $\phi$ maps $F^{d_1}_{d_2}$ onto $\hat{F}^{d_1}_{d_2}$..

Hence by something something ``correspondence of residues'' we have

$$(\ev_\bullet)_*\left(\text{Res}_{F_0^d}(G_{0,d})\right)=\text{Res}_{\hat{F}_0^d}[QG_{0,d}]$$

We know that these are the components of the $\J$ function, but on the right hand side we have one projective space inside a larger projective space, so it is easy to calculate the $\C^*$ action on the normal bundle using the euler exact sequence, giving $\prod\frac{1}{(H+kz)^n}$.


\begin{theorem}
Euler sequence:

$$0\to\OO\to\OO(1)^n\to T_{\proj^{n-1}}\to 0$$
\end{theorem}

\begin{proof}
\emph{Stacks}

Consider $[\C^n\setminus\{0\}/\C^*]\to[\text{pt}/\C^*]=\mathcal{B}\C^*$
and take the triangle relating relative and absoluate cotangent complexes.
\end{proof}

\begin{proof}
\emph{Down-to-earth}

Consider 
$$0\to(\C^n\setminus\{0\}\times\mathfrak{g}\to(\C^n\setminus\{0\})\times\C^n)\to\rho^*T_{\proj^{n-1}}\to 0$$
where the action is the adjoint action.

This is $\C^*$ equivariant, and hence descends to the quotient.
\end{proof}

Whenever the quotient on a vector space we get a similar euler sequence.


Now, to find the normal bundle, there are two euler sequences we must consider, one for the large projective space and one for the smaller projective space.

We take the euler sequence for the target, pull it back to the universal curve, and push it forward, giving us this.

This general process will allow us to get explicit formulas for the residues on many quasimap graph spaces; this is exactly what Givental did when he generalized this from projective space to toric varieties.  This is more complicated in general because it involves a mirror map.

\section{Mirror theorem for the quintic threefold}

Let $X=Z(s), s\in H^0(\proj^4,\OO(5)$, and let $\iota:X\to\proj^4$.  

We will only consider Gromov-Witten invariants where the insertions are pulled back from $\proj^4$, but the intersection form is that of the quintic.

So, a basis would be 
$$\{\one,\iota^*(H),\iota^*(H^2)\iota^*(H^3)\}$$
and the dual basis would be
$$\{\frac{1}{5}\one,\frac{1}{5}\iota^*(H),\frac{1}{5}\iota^*(H^2),\frac{1}{5}\iota^*(H^3)\}$$

Now, let $(C, [u]=[u_1,\dots,u_5])$ be a quasimap to $\proj^4$.  

This is a quasimap to the quintic $X$ if and only if $s(u_1,\dots, u_5)=0$.  
In genus 0, this is saying that the section of a certain bundle on the space of quasimaps to $\proj^4$ vanishes, and so the virtual class of the space of quasimaps to $X$ will be the euler class of that given bundle, and it doesn't matter if we calculate on the space of maps to the quintic or to $\proj^4$.

We have the $\J$ function of the quintic, and another function, call it the $\I$ function, defined by using residues on quasimap spaces.

You can calculate:

$$\I_X(q,z)=\sum q^d \frac{\prod_{i=1}^{5d} (5H+kz)}{\prod{i=1}^d(H+kz)5}$$
$$=\I_0(q)\one+\frac{1}{z}\I_1(q)H+O(\frac{1}{z^2}$$
where
$$\I_0(q)=1+\sum_{d>0} q^d\frac{(5d)!}{(d!)^5}$$

\begin{theorem}
$$e^{\frac{1}{z}\frac{\I_1(q)H}{\I_0(q)}}\J_X(qe^{\frac{\I_1(q)H}{\I_0(q)}},0,z)=\frac{\I_X(q,z)}{\I_0(q)}$$
\end{theorem}

Note: 
$$\frac{\I_1(q)H}{\I_0(q)}\J_X(qe^{\frac{\I_1(q)H}{\I_0(q)}},0,z)=\J_X(q, \frac{\I_1(q)H}{\I_0(q)},z)$$

Three ingredients in Givental's proof

Localization on moduli spaces of maps to $\proj^4$.  He considers the contributions of both sides at the fixed points, and proves that these localizations satisfy two properties.

The first is a recursion relations; on the $\J$ funciton side, this comes from analizying the terms; it is important that there is more than one marking (he uses two).

On the other side, we have unmarked quasimap spaces, and so localization does not give us the same form.  However, we have an explicit form of the $\I$ function, and so we can just check it satisfies the same recursion.

Then, he proves that both sides satisfy a polynomiality property in $z$.  This is geometric on both sides, it comes from writing some potentiual on graph spacesand factoring it using $\C^*$ localization.  It is important that the contributions at $0$ and $\infty$ except the sign of $z$ changes.
v
Finally, he proves a uniqueness lemma that says things that satify polynomiality and recursion agree, if they agree modulo $1/z^2$.  But this is what we can see.

In the next few lectures, we will try to explain how this whole story generalizes.

The point of mirror maps is that we can get a partition function that agrees with the Bmodel side without using the mirror map.  Almost everything agress but the $\Gamma$-class, which this doesn't exactly get.  The get the $\Gamma$ from solutions to the QDEs; for Fano, this is all the same.


\end{document}
