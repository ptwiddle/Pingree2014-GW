\documentclass{amsart}


\linespread{1.2}

\theoremstyle{definition}
\newtheorem{dummy}{}[section]

\newtheorem{theorem}[dummy]{Theorem}
\newtheorem{definition}[dummy]{Definition}
\newtheorem{example}[dummy]{Example}
\newtheorem{question}[dummy]{Question}
\newtheorem{conjecture}[dummy]{Conjecture}
\newtheorem{lemma}[dummy]{Lemma}

\newcommand{\TT}{\mathbb{T}}
\newcommand{\Pic}{\text{Pic}}
\newcommand{\Z}{\mathbb{Z}}
\newcommand{\GIT}{//}
\newcommand{\X}{\mathcal{X}}
\newcommand{\Stab}{\text{Stab}}
\newcommand{\Spec}{\text{Spec}}
\newcommand{\aff}{\text{aff}}
\newcommand{\I}{\mathcal{I}}
\newcommand{\OO}{\mathcal{O}}
\newcommand{\Map}{\textrm{Map}}
\newcommand{\proj}{\mathbb{P}}
\newcommand{\J}{\mathcal{J}}
\newcommand{\sm}{\text{sm}}
\newcommand{\Mbar}{\overline{\mathcal{M}}}
\newcommand{\M}{\mathcal{M}}
\newcommand{\Q}{\mathbb{Q}}
\newcommand{\C}{\mathbb{C}}
\newcommand{\Age}{\text{Age}}
\newcommand{\Fix}{\text{Fix}}
\newcommand{\one}{1}
\newcommand{\st}{\text{st}}
\newcommand{\ttt}{\mathbb{t}}
\newcommand{\ev}{\text{ev}}
\newcommand{\vir}{\text{vir}}
\newcommand{\Quot}{\text{Quot}}
\newcommand{\Hom}{\text{Hom}}


\newcommand{\Aut}{Aut}
\title{Quasimaps and mirror theorems for GIT quotients III}
\author{Ionut Ciocan-Fontanine}


\begin{document}
\maketitle

At the end of the last lecture, we defined quasimap stability.

We had $(W,G,\theta), \theta\in\chi(G)_\mathbb{Q}$.

We could form the affine quotient $X_0$ and stack quotient $\mathcal{X}=[X/G]$, and the GIT quotient $X_\theta==W/GIT_\theta G$.

We defined a $\theta$-quasimap to be a tuple $(C,x,\mathcal{P},u)=(C, x, [u])$ such that $[u]^{-1}(X\setminus X) \text{ finite}$.

We define the class $\beta_{[u]}: \text{Pic}^G W\to \Z$ the numerical class of the map.

\begin{lemma}[Exercise]
$\beta_{[u]}(L_\theta)\geq 0$ and $\beta_{[u]}(L_\theta=0$ if and only $\beta_{[u]}=0$ if and only if $[u]$ is a constant map to $X$.
\end{lemma}

\begin{definition}[Pres-stable]
A $\theta$-quasimap is prestable if the base locus in $C^{ns}$.  In such a case, there exists $[u_{reg}:C\to W\GIT_\theta G$, of class $\beta_{reg}$.
\end{definition}

EXERCISE: $$(\beta-\beta_{reg})(L_\theta)=\sum_{x\in C^{ns}} \ell_theta(x)$$

\begin{definition}[Stable]
A prestable map is $\theta$-stable if
\begin{enumerate}
\item $\ell_\theta(x)+\sum \delta_{x,x_i}\leq 1$ for all $x\in C^{ns}$
\item $\omega_C(\sum x_i)\otimes u^*(\mathcal{P}\times_G)$ is ample.
\end{enumerate}
\end{definition}
We make some comments on this definition.

\begin{itemize}

\item Condition (1) implies that the markings are not base points.
\item Condition (2) implies that 
\begin{enumerate}
\item on every rational bridge $C^\prime\subset C, \beta_{C^\prime}(L_\theta)>0$
\item on every rational tail, $C^{\prime\prime}\subset C, \beta_{{\prime\prime}}(L_\theta)>1$.
\item In particular, if $\beta(L_\theta)\leq 1$, there are no rational tails

\end{enumerate}
\item This definition recores the original $\epsilon$-stability condition with Kim and Maulik as follows.  Let $\theta_0$ be the minimal integral character on the ray generated by $\theta$.  Write $\theta=\epsilon\theta_0$.  Then $\theta$-stability is the same as $\epsilon$-stability with respect to $\theta_0$, and is also equivalent to $\epsilon^\prime=\epsilon/m$ with respect to $\theta^\prime=m\theta_0$.
\item If $k+\beta(C)>0$, then for all $\theta>\theta_0$, a $\theta$-stable quasimap is equivalent to a stable map to $X$.
\item If $(g,k)=(0,0)$, the same is true for $\theta>2\theta_0$.
\item The stability changes at the ``walls'' of $$\theta_0, \frac{1}{2}\theta_0, \frac{1}{3}\theta_0, \dots, \frac{1}{m}\theta_0,\dots$$
For $\theta\in (\frac{1}{m+1}\theta_0,\frac{1}{m}\theta_0]$, the stability condition only allows rational tails of degree $\geq m+1$ with respect to $\theta_0$., and $\ell_{\theta_0}(x)\leq m$ for all $x\in C^{ns}$.
\item
If $\beta_{[u]}$ is fixed, the last wall is $\frac{1}{\beta(L_\theta)}\theta_0$.
\item
In the last chamber, $(0, \frac{1}{\beta(L_{\theta})}\theta_0]$, the stability condition (1) only says that base points are away from markings, and stability condition (2) says that there are no rational tails $[u]^*L_{\theta+0}$ is positive on rational bridges.

We will call such maps $(0+)$-stable quasimaps.  When $X$ is the Grassmannian, this is exctactly the stability condition of Marian, Oprea, and Pandharipande introduced as stable quotients.
\end{itemize}

%Let's order the half ray to be larger away from the origin: $\theta^\prime>\theta^{\prime\prime}$ if $\theta^\prime=\epsilon\theta_0$ and $\theta^{\prime\prime}=\delta\theta_0$, with $\epsilon>\delta$.

The picture we have is what phycists call a Gauged Linear Sigma Model.  We start with a group $G$ acting on a vector space $V$.  We want to study maps to $[V/G]$, this is a bit ugly.
Physicists write a lagrangian involving on many parameters, one of which is the Fayeed something-opolois parameter.  The real part of this is our $\theta$.

At the origin value of $\theta$, we have maps to the artin stack.  When we are ``deep inside the Kahler cone'', we have Gromov-Witten theory of the corresponding target.  If we are very close or very far from the origin, what we have only depends on the chamber, but in between these are some intermediate chabers.

If we look at proof of the Crepant Resolution Conjecture for GIT quotients, we have the wall crossing between $\C^\ell/\Z_\ell$, and $|\OO_{\proj^{\ell-1}}(-\ell)$

They compute the $\J$-function at greater than $\theta_0$, which gives the resolution, and at less than $-\theta_0$, which gives the orbifold.

They show that the $\J$ functions agree with the $I$ functions expanded analytically at $0$ and $\infty$, and hence are analytic continuations of each other.

We claim that this is just a wall crossing problem.  

THe $I$ function we can write simply in terms of the quotient stack $W/G$, it does not depend on the linearlization $\theta$.  So it is conceivable you could try to make sense of it globally on a function.  The semigroup of the Kahler cone changes as we cross from chamber to chamber.

The $\epsilon$-stability was first introduced by Andrei Mustatca, and Toda rediscovered it for the stable quotients.  

\section{Weighted maps}
Let $a=\{a_1,\dots,a_k\}\subset (0,1]\cap \Q$ be weights for the markings.

\begin{definition}
A $\theta$-stable quasimap with weighted markings, of class $\beta$ to $\mathcal{X}$, is an unpointed $\theta^A$-stable quasimap of class $(\beta,1,1,\dots,1)$ to $\mathcal{X}\times[\C/\C^*]^k$.
\end{definition}
 
The same also works for obrifolds, but since we only want to allow orbifold structure of points of weight one, we need to start with a few marked points there.

Depending on the weights, the marked points can come to together, or land in base point.  If we set all the weights to be very small, we get an explicit formula for a big $\J$-function.  This perspective was introduced by Janda.

If we introduce one marking with weight $0+$, we construct the universal curve.

\section{Moduli space}

Fix $(g,k\beta)$, $\M_{g,k}(\mathcal{X},\beta)=\{[u]:(C,x)\to\mathcal{X}, \text{ class } \beta\}$.

Since $\mathcal{X}$ is a global quotient, this is an Artin stack by results of Lieblich.  However, it is rather horrible and ugly Artin stack in general.

For each $\theta\in\chi(G)_\Q$ in the interior of a GIT chamber, we have the open (and usually closed?) substack
$$Q^\theta_{g,k}(X,\beta)=\{[u] \theta\text{-stable}\}$$
This is empty if $2g-2+k+\beta(L_\theta)\leq 0$.

\begin{theorem}[CKM, CCK for orbifolds]
$Q^\theta_{g,k}(X,\beta)$ is a separated DM stack of finite type, proper over $X_0$.

\end{theorem}

In particular, if $X_0$ is projective, this is proper over $\C$.

\subsection{Comments}
\begin{itemize}
\item Seperatedness follows from the prestable condition
\item Finite type needs boundedness of $(\mathcal{P},u)$ (for fixed domain $C$)
\item Properness requires $W$-affine (use Hartog's theorem to extend $u$)


\end{itemize}
No way to extend this to projective $W$; this whole story has no chance of working, because we may need to blow up a basepoint in the central fiber and have an entire component in the unstable locus, giving us basepoints along a whole locus.

EXERCISE: DM comes from stability -- a $\theta$-stable map has finite automorphism group.

\section{Obstruction Theory}
NOTE: HERE, and before, need a fancier script ``M'' to denote the artin stack of curves.

Have a commutative diagram:

$Q^\theta_{g,k}(X,\beta)$ has a map $\mu$ to $\text{Bun}_G$, and a map $\nu$ to $\M_{g,k}$, and there is a map $\phi$ from $\text{Bun}_G$ to $\M$.

We need to look at the $\mu$ relative obstruction theory, governing deformations of sections in the universal family $\mathcal{C}$, which has a map $\pi:\mathcal{C}\to Q_{g,k}(X,\beta)$.  We have $\mathcal{P}$ the universal principle bundle, and the universal section $u:\mathcal{C}\to W\times_G\mathcal{P}$, and $\rho$ going the other way is the projection.

The $\mu$-relative obstruction theory is $*R\pi_* u^* \TT_\rho)^\vee$, where $\TT_\rho$ is the relative tangent bundle.

\begin{theorem}
The obstruction theory is perfect if $W$ has at worst lci intersections (in $W^{us}$).  

In fact, this is an if and only if for $\theta\leq \theta_0$.
\end{theorem}

Some comment about if the singularities are bad, we get infinitely many terms in the cohomology of $\TT_\rho$.  lci guarantees that it's a two term complex.  A theorem of Abramovich says it is either two or infinite, but really three would have been just as bad.

This is why we we made the comment earlier about only complete intersections having a good theory; we can treat any projective variety as a GIT quotient by taking the cone on it, but the singularity at the cone point will be lci if and only if the original projective target was a complete intersection.

\subsection{Remarks}

\begin{itemize}
\item The $\nu$ relative obstruction thoery (over $\M_{g,k}$ is quasi-isomorphic to $(R\pi_X[u]^*\TT_{[W/G]})^\vee$.
\item $W$ lci implies the virtual tangent bundle
$$W\to \text{Va}_{G^{something}}$$ with ideal sheaf $\mathcal{I}$, $[T_V|_W\to (\mathcal{I}/\mathcal{I}^2)^\vee]$, $\TT_\rho\sim\mathcal{P}\times_G T_W$.
\end{itemize}

The virtual dimension is 
$$\text{vdim}=\beta(\text{det} T_W)+(1-g)(\dim X-3)+k$$

The first term should be thought of as
$$\int_{\beta} c_1(T_{[W/G]})$$
the point being that we are using the tangent bundle of the quotient Artin stack, and NOT the GIT quotient.


If $W=V$ is a vector space, then the $\mu$-relative obstruction theory is 
$$R\pi_*(\mathcal{P}\times_G V)$$
We don't even need the Euler sequence, which would express it as the quotient of this by the adjoint lie algebra action.

\section{Invariants}

From now on, to keep the notation the same as that in the references, let $\theta=\epsilon\theta_0$, and write $Q^\epsilon_{g,k}(X,\beta)$, here $\epsilon\in\Q_{>0}\cup\{0+,\infty\}$, with $\infty$ denoting that we are in the GW chamner.

We have evaluation maps
$$\ev_i:Q^\epsilon_{g,k}(X,\beta)\to X$$
(when we go weighted marked points and have marked points with very small weight, they pull back from the quotient stack??)

We also naturally have psi classes

$$\psi_i=C_1(x_i*\omega_{\mathcal{C}})$$
Thus, we have descendent invariants

$$\left\langle \psi^{a_1}_1\gamma_1,\dots,\psi_k^{a_k}\gamma_k\right\rangle^\epsilon_{g,k,\beta}=
\int_{[Q^\epsilon]^\vir} \prod \psi_i^{a_i} \ev_i^*(\gamma_i)$$
where $\gamma_i\in H^*(X)$.

Technically, this only works for $X$-proper, but when we have a torus action with compact fixed points, we can define the above by localization.

\subsection{Novikov ring}
The effective cone of the theory is 
$$\text{Eff}(W,G,\theta)=\{\beta|\beta=\beta_{[u]}\text{for some $\theta$-quasimap } [u]:C\to\mathcal{X}$$
This only depends on the GIT chamber.


In general, this is larger than $NE(X)$.  For example, take $X=\text{pt}=\C\GIT\C^*$, or $\proj^2=\C^4\GIT(\C^*)^2$.

Of course, when we are in the GW theory chamber (or was it the $\proj^2$ chamber?), only one will survive, but in general there will be more.

For example, in the $\proj^2$ case, in many chambers we will have a $ $ variety, and the $\proj^2$ will be the blow-down of 

This is Yunfeng's ``extended fan'' obtained by adding more rays and increasing the degree.

Having the extra parameters is a feature, not a bug; this is what CCIT did with the extended $\J$-function.
$$\Lambda=\Lambda_{W,G,\theta}=\hat{\Q[\text{Eff}(W,G,\theta)]}=\Q[[q]]$$ 


We define 
$$\mathcal{F}_g^\epsilon=\langle\langle - \rangle\rangle^\epsilon_{0,g}$$
and we can also definte the above with $k$.

We have $\mathcal{F}_g^\infty$ the Gromov-Witten potential; the game to play is to figure out what happens with the wall crossing as $\epsilon\to 0$.

\subsection{Observations}

 For $\epsilon\geq 0+$, invariants satisfy splitting.  (missing something about the boundarym being $\epsilon$ sensitive or not?)
In fact, we have $\epsilon$-quasimap classes in $H^*(\Mbar_{g,k})$ using
the stabilization map
$$\st:Q^\epsilon_{g,k}(X,\beta)\to \Mbar_{g,k}$$
and these satisfy splitting.

Thus, we get a CohFT on the same space as GW theory, $(H^*(X),\langle, \rangle)$, and changing $\epsilon$ we have a family of CohFTs.

HOWEVER, in general, the string, divisor, and dilaton equations may fail.  The point is that the universal curve is not the moduli space with one more marking, as soon as we have base points.  As noted, it is the moduli space with one more marking over very small weight. 

This means that the usual proofs of these does not work, and in fact we will see that we can interpret, at least in $g=0$, or in all $g$ for semipositive target, is exactly responsible for the mirror map.  In the case where string, dilaton and everything hold, the mirror map is trivial.


\end{document}
