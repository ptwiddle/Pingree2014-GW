\documentclass{amsart}
\newtheorem{theorem}{Theorem}
\newtheorem{definition}{Definition}
\newtheorem{example}{Example}
\newtheorem{question}{Question}
\newtheorem{conjecture}{Conjecture}

\newcommand{\Mbar}{\overline{\mathcal{M}}}
\newcommand{\M}{\mathcal{M}}
\newcommand{\Q}{\mathbb{Q}}
\newcommand{\C}{\mathbb{C}}
\newcommand{\Age}{\text{Age}}
\newcommand{\Fix}{\text{Fix}}
\newcommand{\one}{1}
\newcommand{\st}{\text{st}}

\newcommand{\Aut}{Aut}
\title{An introducion to FJRW theory}
\author{Tyler Jarvis}

\begin{document}
\maketitle
.Our starting point is 
\begin{theorem}
Intersection theory of $\psi$ classes on $\Mbar_{g,n}$ is governed by the KdV hierarchy
\end{theorem}

Immediately after this, Witten gave two generalizations of $\Mbar_{g,n}$.  The first, are the moduli space of stable maps $\Mbar_{g,n}(X,\beta)$.

The second is $\Mbar^{1/r}_{g,n}$, the moduli space of $r$-spin curves, that is, curves $\mathcal{C}$ together with a line bundle $\mathcal{L}$ so that $\mathcal{L}^{\otimes r}\cong \mathcal{K}_{\log}$.

FJRW is a further generalization of $r$-spin curves, parallel to Gromov-Witten theory.

\begin{tabular}{rcc}
 & $GW$-theory & FJRW \\
\hline
Input & $X$ & $W\in \C[x_1,\dots, x_N]$; $G$ finite \\
State Space & $H^*(X)$ & $\mathcal{A}^{W,G}$ \\
Moduli space & $\Mbar_{g,n}(X,\beta)\to \Mbar_{g,n}$ & $\Mbar_{g,n}^{W,G}\to\Mbar_{g,n}$ \\
Virtual Class & X & X \\
CohFT & \Lambda^{GW} & \Lambda^{FJRW}
\end{tabular}

Larger picture

\begin{tabular}{r|c|c}
 & A & B \\
\hline
CY & GW & BK, BCOV, Costello-Li, etc. \\
\hline
LG & FJRW & Saito-Givental, Dubrovin-Zhang
\end{tabular}

FJRW theory requires the following inputs:
\begin{itemize}
\item  $\C^*$ action on $\C^N$ with weights $c_1,\dots, c_n$; that is $\lambda\cdot(x_1,\dots, x_n)=(\lambda^{c_1}x_1,\dots, \lambda^{c_N}x_N)$.
\item Polynomial $W\in\C[x_1,\dots, x_N]$ that is quasihomogenous of weight $d$ with respect to $\C^*$ action -- that is $W(\lambda\cdot x)=\lambda^d W(x)$.
\item Quasihomogeneity forces $W$ to have a singularity at the origin, we require this singularity to be isolated; that is, $X_W=\{W=0\}\subset WP(c_1,\dots, c_N)$ is smooth.
\item W should have no terms of form $x_ix_j$, which implies that the weights $c_i/d\leq 1/2$ and weights are uniquely determined by $W$.
\end{itemize}

\begin{example}
$W=x^r$ is the $A_{r-1}$ singularity.
\end{example}

\begin{example}
$W=x^5+x^y^2$ is the $D_5$ singularity; we have $d=5, c_1=1, c_2=2$.
\end{example}


The $\C^*$ action and $W$ together give the exponential grading element 
$$J=(\exp(2\pi i c_1/d), \dots, \exp(2\pi i c_N/d)$$
with $J$ fixing $W$.

$G$ is a finite abelian group acting diagonally on $\C^N$ that fixes $W$ and contains $J$.


Let $G^{max}=\{(\alpha_1,\dots,\alpha_N)|W(\alpha_1x_1,\dots,\alpha_Nx_N)=W(x_1,\dots, x_N)\}$ denote the maximal diagonal symmetry group of $W$.

\begin{example}
$$G^{max}_{D_6}=\{\zeta^{-2},\zeta)|\zeta^10=1\}$$
and $J_{D_6}=(\exp(2\pi i /5),\exp(2pi i 2/5))$,
so that $\langle J\rangle \subsetneq G^{max}$.
\end{example}

For all $g\in G^{max}$, we can writ(e $g=(\exp(2\pi i\theta_1),\dots,\exp(2\pi i\theta_N))$, s.t. $\theta_i\in[0,1)\cap \Q$.  We call $\theta_i$ the \emph{phases} of $g$.  

$$\Age(g)=\sum\theta_i$$
$$N_g=\dim \Fix(g)$$

Note: $\Age(g)+\Age(g^{-1})=N-N_g$.


The idea is that, given a the data described above, FJRW theory will produce a CohFT.  What is a CohFT?  It is a collection of tautological classes that behave like you would expect them to if you've been trained in Gromov-Witten theory.



More specifically, we have two gluing maps $\rho_{\text{tree}},\rho_{\text{loop}}$, and the forgetting marked point map $\tau$.


A CohFT with flat identity consists of a state space $\mathcal{A}$, that has a pairing $\langle,\rangle:\mathcal{A}\otimes\mathcal{A}\to\C$, and a family
$\{\Lambda_{g,n}\}$, with $\Lambda_{g,n}\in H^*(\Mbar_{g,n})\otimes (\mathcal{A}^*)^n$
and a vector $\one\in\mathcal{A}$ that behave well with respect to the gluing maps, that is:

$$\rho^*_{\text{tree}}\Lambda_{g_1+g_2, k_1+k_2}(\alpha_{k_1},\dots,\alpha_{k_1+k_2})
=\sum_{r,s} \Lambda_{g_1, k_1+1}(\alpha_1,\dots,\alpha_{k_1},r)\eta^{rs}\Lambda_{g_2, k_2+1}(s,\alpha_{k_1+1},\dots,\alpha_{k_1+k_2})$$
here $r,s$ run over a basis of $\mathcal{A}$, and $(\eta^{rs})$ is the inverse of the metric $\langle, rangle$ in terms of a basis of $\mathcal{A}$.

We have analogous requirements for $\rho^*_{\text{loop}}$:
$$\rho^*_{\text{loop}}\Lambda_{g+1,k}(\alpha_1,\dots,\alpha_k)=\sum_{r,s}\Lambda_{g,k+2})(\alpha_1,\dots,\alpha_k,r,s)\eta^{rs}$$

We want $\Lambda_{g,k}$ to be symmetric; that is, $S_k$ invariant, as an element of $H^*(\Mbar_{g,n})\otimes(\mathcal{A}^*)^k$.

The flat identity condition states that y

$$\tau^*(\Lambda_{g,k}(\alpha_1,\dots,\alpha_k)=\Lambda_{g,k+1}(\alpha_1,\dots,\alpha_k,\one)$$
and $\Lambda_{0,3}(\alpha_1,\alpha_2,\one)=\langle \alpha_1,\alpha_2\rangle$.

To construct the CohFT, we will use a moduli space (stack) consisting of curves with with extra structure.

The moduli space will be $\Mbar_{g,k}^{W,G}$ with maps $e_i$ to $\mathcal{I
B}G$, and a finite stabilization map $\st$ to $\Mbar_{g,k}$.

The moduli space $\Mbar_{g,k}^{W,G}$  is smooth and proper.


This will be based off 
$$\mathcal{W}_{g,k}^{W,G}=\{\mathcal{C},p_1,\dots,p_k,\mathcal{L}_1,\dots,\mathcal{L}_N,\phi_i\}$$
where
\begin{itemize}
\item $(\mathcal{C},p_1,\dots, p_k)$ is a stable pointed orbicurve.
\item $\mathcal{L}_i$ are orbifold line bundles on $\mathcal{C}$
\item $W=\sum{i=1}^s a_iW_i$, with $w_i$ monomials in the $x_i$
\item $\phi_i:W_i(\mathcal{L_1},\dots,\mathcal{L}_N)\to\mathcal{K}_{\log}$
\item The group action at each marked point and node factors through $G$.
\end{itemize}

Here $\mathcal{K}_\log=\mathcal{K}\otimes\mathcal{O}(\sum p_i)$ has sections locally given by $dz/z$ around the marked points.

\begin{example}{$r$-spin curves}
Let $W=x^r, J=\exp(2\pi i/r), G^{\text{max}}=\langle J\rangle$
and 
$$\Mbar_{g,k}^{x^r,G^{\text{max}}}={(\mathcal{C},p_1,\dots,p_k $$

Polischchuck-Vaintrob have an alternate definition.

Let $\Gamma=\C^*G\subset GL(N)$
MORE IN NOTES

\section{State Space}
First we will give the high-brow definition; then we will give a more down-to-earth version.

$W:\C^N\to\C$ is $G$-invariant, and so defines a map
$W:[\C^N/G]\tp\C$.
-
$W^\infty=\text{Re}W^{-1}\left((M,\infty)\right), M>>0$;
that is, we want to consider those points in $[\C^N/G]$ that map to something with very large real part under $W$.
Then:
$$\mathcal{A}=H_{CR}^{*+2\sum c_i/d} ([\C^N/G],W^\infty,\C)$$
In another way, this is 
$$\bigoplus_{\gamma\in G} H^*(\Fix(\gamma);W_\gamma^\infty,\C)^G$$

Note that $\Fix(\gamma)=\C^{N_\gamma}$.

If $\gamma=J=(\exp(2\pi i c_1/d),\dots,\exp(2\pi i c_N/d))$, then $\Fix(J)=\{0\}$, and $\mathcal{A}_J=H^*(\{0\},\C)=\C$, and $\one\in\mathcal{A}$ is defined to be $1\in\mathcal{A}_J$.

Note that this is a bit surprising for those used to Chen-Ruan cohomology, as the identity lives in a twisted sector.

Another, more computational description:

\begin{theorem}{Wall, Orlik-Solomon, Sebastrani}
$$H^*(\C^n, W^\infty;\C)\cong\Omega^N/(dW\wedge\Omega^{N-1}$$ as $G$-modules.
\end{theorem}

This is germs of $N$-forms on $\C^N$ at 0, modulo $dW=0$.

Another way to put this is $\C[x_1,\dots,x_N]dx_1\wedge\cdots\wedge dx_N/(\frac{\partial W}{\partial x_1}dx_1+\cdots+\frac{\partial W}{\partial x_N}dx_N)\wedge dx_1\wedge\cdots\wedg dx_N$.

$\cong \C[x_1,\dots,x_N]/(\frac{\partial W}{\partial x_1},\dots, \frac{\partial W}{\partial x_N})dx_1\wedge\cdots\wedge dx_N$
THis is the Milnor ring, or local algebra, of $W$.  

Fact: $\Omega^N/dW\wedge\Omega^{N-1}$ is finite dimneionsal ifa nd only if $W$ has an isolated singularty at 0.  The dimension is called the milnor number, and is denoted $\mu$.  

The Milnor ring is powerful: for quasihomogenous singularities, that is a natural class of equivalence of singularities, so that two singularities are equivalent if and only if their Milnor rings are isomorphic.


The pairing on $H^*(\C^N, W^\infty,\C)$ matches the residue pairing on $\Omega^N/dW\wedge\Omega^{N-1}$.  

Residue pairing:

$$\langle fdx_1\wedge\dots\wedge dx_N,gdx_1\wedge\dots\wedge dx_n\rangle=
\frac{1}{(2\pi i)^N}\int \frac{fg dx_1\wedge\dots\wedge dx_n}{\frac{\partial W}{\partial x_1}\cdots\frac{\partial W}{\partial x_N}}$$


Fact: in $\C[x_1,\dots,x_N]/(frac{\partial W}{\partial x_1},\dots,\frac{\partial W}{\partial x_n})$ there is an obvious gradiing.  The Hessian of W spans the part of highest degree.  If $\langle f,g\rangle=\alpha\mu$, then $fg=\alpha \text{Hess}+\text{lower order terms}$.


Fact: $$H^{N-p,p}(\C^n,W^\infty. \C)^{\langle J\rangle}\to\{\phi dx_1\wedge\dots dx_N|\deg \phi=p\}$$

Highest degree in Milnor ring is the degree of the Hessian, which is denoted $\hat{c}=\sum (1-2(c_i/d))$.

Then $\langle,\rangle:\mathcal{A}^{p,q}\otimes\mathcal{A}^{\hat{c}-p,\hat{c}-q}\to\C$.

So that $\mathcal{A}$ behaves like the cohomology of a manifold of dimension $\hat{c}$, which may be fractional -- $\mathcal{A}$ may have fractional grading anyway.  Bu
 if $\hat{c}=N-2$, then $X_W=\{W=0\}$ is CY, and $\mathcal{A}$ agrees with the cohomology of $X_W$.

MORE ON BIGRADING LATER

We have a moduli space, and a state space, the idea is that we will construct a virtual fundamental class $[\Mbar_{g,k}^{W,G}]^\vir\in H_*(\Mbar^{W,G}_{g,l})\otimes (\mathcal{A}^*)^k$ such that the classes we get by pushing down the virtual classes to $\Mbar_{g,k}$ we get a CohFT.

More specifically, define:

$$\Lambda_{g,k}=\frac{PD \st_*[\Mbar_{g,k}^{W,G}]^\vir}{|G|^g\deg\st}$$

$\Lambda_{g,k}$ forms a CohFT

$\dim \Lambda_{g,k} (\alpha_1,\dots,\alpha_k)=(\hat{c}-3)(1-g)+k-\sum\deg\alpha_i$
G^{\text{max}}-invariant.

Deformation invariance: If $W_t$ is a 1-parameter family of polynomials, then $\Lambda_{g,k}^{W_t,G}$ is independent of $t$.

Decomposition: $\Lambda^{W_1+W_2,G_1\times G_2}_{g,k}=\Lambda^{W_1,G_1}_{g,k}\otimes\Lambda^{W_2,G_2}_{g,k}$.

Additional properties that facilitate computation.



\end{document}
