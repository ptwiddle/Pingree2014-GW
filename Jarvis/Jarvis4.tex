\documentclass{amsart}

\linespread{1.2}
\theoremstyle{definition}
\newtheorem{dummy}{}[section]
\newtheorem{theorem}[dummy]{Theorem}
\newtheorem{definition}[dummy]{Definition}
\newtheorem{example}[dummy]{Example}
\newtheorem{question}[dummy]{Question}
\newtheorem{conjecture}[dummy]{Conjecture}

\newcommand{\vir}{\text{vir}}
\newcommand{\Z}{\mathbb{Z}}
\newcommand{\ev}{\text{ev}}
\newcommand{\Mbar}{\overline{\mathcal{M}}}
\newcommand{\M}{\mathcal{M}}
\newcommand{\Q}{\mathbb{Q}}
\newcommand{\C}{\mathbb{C}}
\newcommand{\Age}{\text{Age}}
\newcommand{\Fix}{\text{Fix}}
\newcommand{\one}{1}
\newcommand{\st}{\text{st}}

\newcommand{\Aut}{Aut}
\title{An introducion to FJRW theory IV}
\author{Tyler Jarvis}

\begin{document}
\maketitle
Today, Nathan Priddis is substituting in as Tyler is losing his voice.

The plan today is to talk about two things: LG mirror symmetry, and the LG/CY correspondence.

\section{LG mirror symmetry}

First, we will review what we have done

\subsection{A-model review}


\subsubsection{Input}
Recall we started with:


A $W$-quasihomogeneous polynomial satisfying:
\begin{itemize}
\item weights $q_i=c_i/d$
\item Isolated singularity 
\item No terms of the form $x_ix_j$
\end{itemize}

and a group of symmetries $G^W$:


 $$G^W=\left\{(\alpha_1,\dots,\alpha_N)\in(\C^*)^N|W(\alpha_1x_1,\dots,\alpha_Nx_n)=W(x_1,\dots, x_N)\right\}$$

With the grading element $J_W$.

$G^W$ is what we were calling $G^{max}$ before, but later we will have two maximal groups (one from the mirror).

\subsubsection{Output}

From this, we first constructed the state space

\begin{align*}
A^{W,G}&=H^*_{CR}([\C^N/G], W^\infty,\C) \\
&=\bigoplus_{g\in G} A_g
\end{align*}

The grading was

$$(p+\sum_{i=1}^N \Theta_i^g-\sum_{j=1}^N q_i, N_g-p+\sum_{i=1}^N \Theta_i^g-\sum_{j=1}^N q_i)$$

Here $p$ is the internal weight, each piece was a Jacobi ring, and this was the weight coming from that.  The $\Theta$ terms are the weights, similar to the age in Chen-RUan cohomology, and the $q_i$ are an extra shifting, which is the age of $J$ -- recall, that our identity lived in $A_J$, and so the shifting of $A_J$ should be trivial.

We had the moduli space of $W$ structures, with a virtual fundamental class that we could push forward to $\Mbar_{g,n}$ to obtain a CohFT:

$$\Lambda_{g,n}(\phi_1,\dots,\phi_n)\in H^*(\Mbar_{g,n})$$

Integrating these classes over $\Mbar_{g,n}$, we get numbers, called correlators, which are the FJRW invariants.

\subsection{B-model}

We now turn to the $B$-model.  The input to the $B$ model is the same as the $A$-model; a quasihomogeneous polynomial $W$ and a finite group of symmetries $G\subset G^W$.



We have already seen some of the ingredients that go into the $B$-model.

The main piece is $\Omega_W=\Omega^N/\partial W\wedge \Omega^{N-1}=\C[x_1,\dots, x_N]/(\partial_1 W,\dots, \partial_N W)$

$\Omega_W$ is called the Jacobi ring (or Milnor ring?)

The fact that $W$ had an isolated singularity at the origin means that $\Omega_W$ is finite dimensional, with dimension

$$\mu=\prod_{j=1}^N (\frac{1}{q_j}-1)$$



It has a top degree element called the Hessian 
$$\text{Hess}(W)=\text{det}\left(\partial_i\partial_j W\right)$$

\subsubsection{The role of $G$}
The group $G^W$ acts on $\Omega_W$:

\begin{align*}
(\alpha_1,\dots,\alpha_n)\cdot\prod_{j=1}^N x_j^{m_j}dx_1\wedge\dots\wedge dx_N
=\prod_{j=1}^N \alpha_j^{m_j+1}\prod_{j=1}^N x_j^{m_j}dx_1\wedge\dots\wedge dx_N
\end{align*}

\subsubsection{Pairing}
The pairing is defined by taking the coefficient of the top dimensional part in $fg$, and normalizing by $\mu(W)$:

$$fg=\frac{\langle f, g\rangle}{\mu(W)}\cdot\text{hess(W)}+ \text{ lower order}$$
From this, it is clear that
$$\langle fg,h\rangle=\langle f, gh\rangle$$
and so this structure gives us a Frobenius algebra.


\subsubsection{Groups}

On the $A$ side, our group $G$ had to be ``admissible'', which means it contains $J$.  Dual to that on the $B$ side, we have that $G$ must be contained in $SL_W=SL_N\cap G^W$.

Our state space will be

$$\mathcal{Q}_{W,G}=\bigoplus_{g\in G} \left(\Omega_{W_g}\right)^G$$
Our grading is given by:

$$(p+\sum_{i=1}^N \Theta_i^g-\sum_{j=1}^N q_i, p+\sum_{i=1}^N \Theta_i^g-\sum_{j=1}^N q_i)$$
which is similar to the grading on the $A$ model, but symmetric instead of antisymmetric.

The pairing on the orbifolded state space is constructed similarly to before:

$$\langle,\rangle:\Omega_{w_g}\otimes\Omega_{w_{g^{-1}}}\to\C$$
which gives us a Frobenius algebra.

There is an obvious isomorphism $A_{W,G}\cong \mathcal{Q}_{W,G}$, as vector spaces, but the gradings and pairings are different.

\subsection{Mirror Symmetry}

\begin{definition}
A quasihomogeneous polynomial is \emph{invertible} if it has the same number of monomials and variables, (and it satisfies the properties it had before)
\end{definition}

So $W=\sum_{i=1}^N\prod_{j=1}^N x_j^{a_{ij}}$.

\begin{theorem}[Kreuzer, Skarke]
Let $W$ be an invertible polynomial satisfying our main conditions.

Then $W$ is the disjoint sum of the following atomic types:
\begin{enumerate}
\item Fermat: $x^a, a\geq 2$
\item Chain: $x_1^{a_1}x_2+x_x^{a_2}x_3+\cdots x_{N-1}^{a_{N-1}}x_N+x_N^{a_N}$ with $a_i\geq 2$
\item Loop: $x_1^{a_1}x_2+x_x^{a_2}x_3+\cdots x_{N-1}^{a_{N-1}}x_N+x_N^{a_N}x_1$ with $a_i\geq 2$
\end{enumerate}

\end{theorem}

The exponent matrix is $E_W=(a_{ij})$.

The inverse matrix $E_w^{-1}=(a^{ij})$. 

Define $\rho_j\in(\C^*)^N$

$$\rho_j=\left(\exp(2\pi i a^{1j}),\dots,\exp(2\pi i a^{Nj})\right)$$

FACT 1:
Then $\{\rho_j\}$ generated $G^W$.  This is not at all a minimal generating set, but will be an important set of generators that will let us write the dual group.  

We leave the proof as an exercise, but the hint it $E_W^{-1} E_W=\text{Id}$.

FACT 2: $q_i=\sum_{j=1}^N a^{ij}$

This implies that $J=\rho_1\cdots\rho_N$.

\begin{example}[$E_7=x^3+xy^3$]

We have

$$E_w=\begin{pmatrix} 3 & 0 \\ 1 & 3 
\end{pmatrix}$$

$$E^{-1}_w=\begin{pmatrix} \frac{1}{3} & 0 \\ \frac{-1}{9} &\frac{1}{3} 
\end{pmatrix} $$

$$J=\left(e^{2\pi i/3}, e^{2\pi i 2/9}\right)$$
\end{example}


\begin{definition}[Berglund, H\"ubsch, Krawitz, Henningson]
\begin{enumerate}
\item The dual polynomial to $W$, written $W^T$, is given by $E_W^T$.
\item The dual grop $G^T$, is
\begin{align*}
G^T&=\left\{ \overline{\rho}_1^{b_1}\cdots \overline{\rho}_N^{b_N} | \prod_{j=1}^N x_j^{b_j} \text{ is $G$-invariant}\right\} \\
&=\left\{ \overline{\rho}_1^{b_1}\cdots \overline{\rho}_N^{b_N} | (b_1,\dots, b_n)E_w^{-1} (\ell_1,\dots,\ell_n)^T\in\Z \text{ for all } \rho_1^{\ell_1}\cdots\rho_N^{\ell_N}\in G\right\}
\end{align*}
\end{enumerate}

\end{definition}

Facts:
\begin{enumerate}
\item $(G^T)^T=G$
\item \begin{align*}
\langle J_W\rangle^T&=SL_{W^T}\\
SL_W^T&=\langle J_{W^T}\rangle 
\end{align*}
\item $G_1<G_2$ if and only if $G_2^T<G_1^T$
\item $(G^W)^T=\{ 1\}$
\end{enumerate}

\begin{example}[$E_7^T$]
We had:
$$E_w=\begin{pmatrix} 3 & 0 \\ 1 & 3 
\end{pmatrix}$$
So
$$E_w^T=\begin{pmatrix} 3 & 1 \\ 0 & 3 
\end{pmatrix}$$
and hence we see
$$E_y^T=x^3y_y^3$$
and so $E_7$ is self-dual.
\end{example}

\begin{conjecture}
The LG A-model for $(W,G)$ is mirror to the LG B-model for $(W^T, G^T)$.
\end{conjecture}

\begin{theorem}[Krawitz]
There is an isomorphism 
$$A_{w,G}\cong \mathcal{Q}_{W^T, G^T}$$
that preserves bi-degrees and the pairing.
\end{theorem}
This can be viewed as a state-space isomorphism.  We do not know an isomorphism of Frobenius algebras for all $(W,G)$ yet.

Furthermore, if $W$ gave a CY in weighted projective space, we could look at the hodge diamonds of $X_W/G_W$ and $X_{W^T}/G_{W^T}$, and they would be related by the familiar 90 degree rotation.

\begin{proof}
Look at the vector spaces
$$\bigoplus_{g\in G^W} \Omega_{W_g} \quad\text{and}\quad \bigoplus_{h\in G^{WT}}\Omega_{W_h^T}$$
this is like our state space, but we are summing over all of $G^W$, and not taking invariants.

There is an isomorphism between these spaces given by $(v,g)\mapsto (w, h)$ as:
$$\left(\prod_{j=1}^{N_g}x_{i_j}^{b_{i_j}}d\underline{x},\prod_{r=1}^{N_h} \rho_{i_r}^{s_{i_r}+1}\right)\mapsto \left(\prod_{r=1}^{N_h} y_{i_r}^{s_{i_r}}d\underline{y},\prod_{j=1}^{N_g}\overline{\rho}_{i_j}^{b_{i_j}+1}\right)$$
\end{proof}

\subsection{Frobenius algebras}

Several cases of isomorphisms of Frobenius algebras have been proven, but not the general case.  We have
\begin{itemize}
\item Krawitz showed we could take the maximal group:
$$A_{W,G^W}\cong\mathcal{Q}_{W^T,\one}$$
\item Francis, Jarvis, Johnson and Webb for all $G$ if $W$ is a fermat+loops, and $W$ is a chain, for some groups.
\end{itemize}

Note: the restrictions on the groups is because even if the polynomial decomposes as a sum of basic pieces, its group of symmetries may or may not also decompose.

\subsection{Frobenius Manifolds}
In the first FJR paper, they proved an isomorphism of Frobenius manifolds for $W$ an ADE case with $\hat{c}<1$.  The $A$ case had been proven earlier by Faber-Shadrin-Zvonkine.

Krawitz, Shen, Milanov proved an isomoprhism of Frobenius manifolds for simple elliptic singularities $(\hat{c}=1)$ and maximal symmetry group on the A-side.

Li-Li-Saito-Shen proved the isomorphism of singularities for the 14 exceptional singularities in Arnold's classification and maximal symmetry group on the A-side.

He-Webb-Shen-Li: have the isomorphism of Frobenius manifolds for all $W$ and maximal symmetry group on the A-side.

One reason they use maximal symmetry group on the A-side is that there is still some work to be done to understand orbifolding the B-side.

\subsection{J-functions and I-functions}
Another flavor of result is to show an equality of small $J$ and $I$ functions.


Chiodo-Ruan: $W=\sum_{i=1}^5 x_i^5, G=\langle J\rangle$

Priddis-Shoemaker: $W=\sum_{i=1}^5, G=SL_W$

Gru\'er\'e:$W$ is a chain, $G=\langle J\rangle$


Most of these results are just in genus 0.

The first series of results actually hold for higher genus, because for $G^{max}$ the B-model Frobenius manifold is generically semisimple, and so we can get the higher genus results using Teleman's result.  There is still some difficulty as there is a shift that gives an infinite sum, and work is necessary to show that this infinite sum converges.

\section{LG-CY correspondence}
Not much time, but we'll give just a little bit of the flavor.

In the same Kahler moduli space that corresponds to FJRW $(W,G)$, there will be a large-radius limit point (maximal unipotent monomdromy), that corresponds to the Gromov-Witten theory of $X_W/(G/J)$.  

The $J$-function is function that lives on the Kahler moduli space.

Mirror to this, we have the $I$ function on the $B$-model moduli space.

So, we use GW mirror symmetry, analytically continue the $I$-funciton on the B-model moduli space and apply a symplectic transformation to get the $I$-function at the orbifold point.  Another mirror map should give the J-function of the FJRW theory of $(W,G)$.

One of the main motivations for FJRW theory was to help compute GW-theory.  The hope is that it is easier to compute, and then we can apply the network of mirror symmetry, analytic continuation and symplectic transformations to find the Gromov-Witten theory.
\end{document}

