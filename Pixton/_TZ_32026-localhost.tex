\documentclass{amsart}

\linespread{1.2}

\theoremstyle{definition}
\newtheorem{theorem}{Theorem}
\newtheorem{definition}{Definition}
\newtheorem{lemma}{Lemma}
\newtheorem{example}{Example}
\newtheorem{question}{Question}
\newtheorem{conjecture}{Conjecture}
\newtheorem{corollary}{Corollary}

\newcommand{\proj}{\mathbb{P}}
\newcommand{\C}{\mathbb{C}}
\newcommand{\one}{\text{one}}
\newcommand{\End}{\text{End}}
\newcommand{\Mbar}{\overline{\mathcal{M}}}
\newcommand{\M}{\mathcal{M}}
\newcommand{\Q}{\mathbb{Q}}

\newcommand{\Aut}{Aut}


\title{The tautological ring and cohomological field theories IV}
\author{Aaron Pixton}
\begin{document}
 \begin{multline*}
C^r_{g,A}=\sum_{\Gamma \text{graph}, w:H(\Gamma)}\frac{1}{|\Aut(\Gamma)|r^{|h_1(\Gamma)|}\\
 \iota_{\Gamma*}\left( \prod_{\ell} \exp(w(\ell)^2\psi_\ell) \prod_{(h,h^\prime)} \frac{1-\exp(-w(h)w(h^\prime)(\psi+\psi^\prime))}{\psi+\psi^\prime}\right) 
\end{multline*}
where now the weight $w$ takes values in $\{0,1,\dots,r\}$, and satisfies the restraints we had before, but now only mod $r$.


\begin{theorem}
For $r$ sufficiently large, $C_{g,A}^r$ is polynomial in $r$.
\end{theorem}

\begin{definition}
$$C_{g,A}=C^r_{g,A}|_{r=0}$$
\end{definition}

\begin{conjecture}


\end{document}
