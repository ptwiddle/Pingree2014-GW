\documentclass{amsart}
\newtheorem{theorem}{Theorem}
\newtheorem{definition}{Definition}
\newtheorem{example}{Example}
\newtheorem{question}{Question}
\newtheorem{conjecture}{Conjecture}

\newcommand{\Mbar}{\overline{\mathcal{M}}}
\newcommand{\M}{\mathcal{M}}
\newcommand{\Q}{\mathbb{Q}}

\newcommand{\Aut}{Aut}


\title{The tautological ring and cohomological field theories I}
\author{Aaron Pixton}
\begin{document}
\maketitle

Here's the brief outline

\begin{enumerate}
\item Introduction to the tautological ring
\item Cohomological field theories and Witten's $r$-spin class (two talks)
\item Recent conjectures on double ramification cycles
\end{enumerate}

The general theme of the talks is that we want to do intersection theory on $\Mbar_{g,n}$; to connect this to Gromov-Witten theory, we can observe that this is the moduli space of stable maps to a point.

Intersection theory means we want to consider the Chow ring $A^*(\Mbar_{g,n})$; this is huge and ugly, even just for $\M$.  To fix this, we restrict to the smallest subring containing the classes we want to talk about, this is the \emph{tautological ring} $R^*(\Mbar_{g,n})\subset A^*(\Mbar_{g,n})$.  We will give two definitions now, but to start out, you can think of the tautological ring as containing the geometrically interesting classes.

\begin{definition} The tautological rings $R^*(\Mbar_{g,n})$ are defined simultaneously for all $g,n$ as the smallest subrgins (wth unit; we don't believe in rings that do not have a unit) of $A^*(\Mbar_{g,n})$ that are
\begin{itemize}
\item Closed under pushforward along the various maps between moduli spaces -- i.e., forgetting points and gluing points

\end{itemize}

\end{definition}


Note: we are being slightly sloppy here; whenever we are talking about subring, we really mean sub $\Q$-algebra.  
So, classes in the tautological rings are ones coming from the natural maps between moduli space.

\subsection{Another view: what do tautological classes actually look like}

Here are some natural classes in the tautological ring

\subsubsection{Psi classes}
The space $\Mbar_{g,n}$ comes with $n$ associated line bundles, $\mathbb{L}_1,\dots, \mathbb{L_n}$.  These are best described by giving the fiber over a curve $\mathcal{C}$ -- the fiber over $\mathbb{L}_i$ is given by $T^*_{p_i}C$; alternatively, we may use the dualizing sheaf $\omega_C$.

$$\psi_i=c_1(\mathbb{L}_i)$$

It is not immediately clear that $\psi_i$ lies inside the tautological ring.

\subsubsection{Kappa classes}

Consider the forgetful map $\pi:\Mbar_{g,n+1}\to\Mbar_{g,n}$ forgetting the last point.

$$\kappa_m=\pi_*(\psi_{n+1}^m+1)\in A^m(\Mbar_{g,n})$$
\subsubsection{Boundary strata}
Given any $[C,p_1,\dots,p_n]\in\Mbar_{g,n}$, its dual graph is given as follows.  The irreducible components of $C$ correspond to the vertices of $\Gamma_C$, labeled by $g$, the nodes in $C$ correspond to the edges of $\Gamma_C$, and marked points of $C$ correspond to half edges (or legs) of $\Gamma_C$.
EXAMPLE

Given any graph $\Gamma$, we obtain a tautological class %$[\overline{\{\text{curves with dual graph $=\Gamma$}\}]$

Given such a $\Gamma$, we obtain a generalized gluing map $\iota_\Gamma$ whose immage is precisely the locus.   

One can see that the boundary locus is $\frac{1}{|\Aut(\Gamma)|}\iota_{\Gamma*}(1)$, and hence in the tautological ring.



Fun exercise: how do you see that $\psi$ classes are in the tautological ring?


\begin{theorem}[Graber-Pandharipande]

$R^*(\Mbar_{g,n})$ is additively generated by classes $\iota_{\Gamma*}(m)$, where $m$ is any monomial in $\kappa$ and $\psi$ classes.
\end{theorem}

Note: this theorem was probably known before, but was first written down by GP.  The content of the theorem is giving a complicated but explicit algorithim for expressing the product of any two of these classes as a linear combination of these classes.

The central question is to know what the tautological ring looks like.  GP gives us a set of generators of the ring; thus, we need to know relations between these generators.

\subsection{Examples of relations}

\begin{example}[WDVV]
On $\Mbar_{0,4}$, we have $[12|34]=[13|24]=[14|23]$.
\end{example}


\begin{example}
In $R^2(\Mbar_{1,4})$, there are $9$ boundary strata up to the $S_4$ action permuting the four markings.  Let $\bullet$ denote a genus 1 curve, and $\circ$ a genus 2 curve.  


Getzle noticed that these nine strata generate a 7 dimensional vector space; so there are two relations between them.  One of these relations follows from WDVV, involving the 5 classes that do not involve a genus 1 curve, obtained by pushing forward the WDVV equation along the map gluing two points together.

However, there is one other relation that does not come from WDVV.    
\end{example}

This shows that these already become quite complicated; it is not clear where Getzler's relation comes from; he calculated it rather indirectly.

In genus 1, these are essentially the only relations; we don't know in general how many relations we need to find everything.

\subsection{Tautological relations}
Such relations are very useful, because we can pull them back to get recursion relations in Gromov-Witten theory.

$$\Mbar_{g,n}(X,\beta)\to\Mbar_{g,n}$$
and relations on $\Mbar_{g,n}$ pull back.  If we consider the classes that do not involve $\psi$ and $\kappa$ classes, these are relatively easy to understand; if we include $\psi$ and $\kappa$ classes; e.g., descendent classes, it is much harder.  But it has been essentially carried out in genus 0 and genus 1, and used to prove the Virasoro conjectures here.

\section{Switching gears to $\M_g$ }
Including the whole tautological ring is very complicated, but there is already a very rich theory if we only consider smooth curves.

\begin{definition}
If $M$ is an open subset of $\Mbar_{g,n}$, we define $R^*(M)$ as the image under restriction of $R^*(\Mbar_{g,n})$.
\end{definition}

\begin{example}
$R^*(\Mbar_g)$ is the subring of $A^*(\Mbar_g)$ generated by $\kappa_1,\kappa_2,\dots$.
\end{example}

Thus, understanding the tautological ring of $\M$ consists of understanding which polynomials in the kappa classes vanish.

Faber's conjecture:
\begin{enumerate}
\item $R^d(\M_g)=0$ for $d>g-2$
\item $R^{g-2}(\M_g)=\Q$, with an explicit isomorphism
\item $R^d(\M_g)\times R^{g-2-d}(\M_g)\to R^{g-2}(\M_g)=\Q$ is a perfect pairing.
\end{enumerate}
Faber's conjecture's may be summarized as stating that the cohomology of $\M_g$ behaves like the cohomology of a $g-2$ dimensional compact manifold.

Also, this conjecture completely determines the structure of the tautological ring. 

These say that the ring is a gorenstein ring with a one dimensional socle in dimension $g-2$.


Parts 1 and 2 are known.  Part 3 is true for $g\leq 23$, unknown for all $g\geq 24$.  The first two parts were proven fairly quickly after the conjecture was announced in the 90's.  

How is it known up to genus 23?  Faber did this with computer calculation.  There are certain conjectures that are predicted to be true, using very classical methods of computing relations.  If you can use these relations to find all the relations you need, you're done.  What happens in $g\geq 24$, is that these classical methods of producing relations fails to find all the relations.  In particular, when $g=24$, there is exactly one missing relation in $R^12(\M_24)$, not found by Faber's methods.  So, there is an explicit polynomial in the $\kappa$ classes, that we don't know if it is a relation or not.

However, most people probably believe that the conjecture is false.

\section{Faber-Zagier relations}
The FZ relations are a combinatorial description of finitely many relations in each $R^d(\M_g)$, parametrized by partitions $\sigma$ satisfying:


The FZ relations span all known relations, and agree with other constructions of relations.  So, for example, the FZ relations give exactly the known relations in genus 24.

\begin{conjecture}
The FZ relations span all relations in $R^*(\M_g)$.
\end{conjecture}

The FZ conjecture was not made by Faber or Zagier.

For $g\leq 23$, both the Gorenstein and FZ conjectures hold.  For $g=24$, exactly one of them is true, for $g>24$, at most one is true.

\subsection{Other evidence for Gorenstein conjecture being false}
There is a version of Faber's conjectures for $\Mbar_{g,n}$, obtained by changing the dimension of the socle to $3g-3+n$.  This version of the conjuctures has been shown to be false for $\Mbar_{2,20}$ by Petersen-Tommasi.  So, perhaps maybe we should forget about Gorenstein, and focus on Faber-Zagier relations.

Also, if we want to consider $\M_{g,n}$, the proper thing to consider is rational tails; it is also not known if the Gorenstein conjecture holds here.

We're going ot describe the Faber-Zagier relations; this will be a prelude to an extension of them to $\Mbar_{g,n}$.

Which partitions $\sigma$ do we consider to get relations?

We want that $\sigma$ has no parts congruent to 2 mod 3, $3d\geq g+|\sigma|+1$, and $3d$ congruent to $g+|\sigma|+1\ mod 2$.

Faber-Zagier guessed these conditions from numerical data.

\begin{definition}
$$A(T)=\sum_{n\geq 0} \frac{(6n)!}{(3n)!(2n)!} T^n$$
$$A(T)=\sum_{n\geq 0} \frac{6n+1}{6n-1}\frac{(6n)!}{(3n)!(2n)!} T^n$$

\end{definition}

To get $\kappa$ classes in, we will use $\{\sum c_i T^i\}=\sum c_i\kappa_i T^i$.
$$FZ(g,d,\sigma)=[\exp(-\{\log(A(T)(1+p_3T+p_6T^2+\cdots)+B(T)(p_1+p_4T+p_7T^2+\cdots))\})]_{p^\sigma T^d}$$

Note that the exp and log don't cancel because we have inserted the $\kappa$ classes in between.

Note that this will have $\kappa_0$ appearing; we define $\kappa_0=2g-2+n$.  

\begin{theorem}[Pandharipande-Pixton]
$FZ(G,d,\sigma)=0$ for $\sigma$ satisfying the conditions described.
\end{theorem}
Faber and Zagier had conjectured these relations on numerical evidence, but hadn't proven it, and hadn't taken it that seriously because it contradicted the gorenstein conjecture.


One might imagine that adding in boudnary locus would make this more complicated; it makes it simpler in some ways.

One problem with the FZ relations is that it's hard to see how the different ones are related to each other; some aspects of this will become simpler for $\Mbar_{g,n}$.

Instead of our relations being parametrized by $g,d,\sigma$, our relations will just be parametrized by $R(g,n,d)$ will be a relation in $R^d(\Mbar_{g,n})$ if $3d\geq g+n+1$, and has the same parity.  So we have just one relation in each place; these will somehow generate all known relations.

$$R(g,n,d)=\left[\sum_{\Gamma} \frac{T^{|E(\Gamma)|}}{\Aut(\Gamma)}\iota_{\Gamma*}\left( \prod_{v\in\Gamma} f_v\prod_{e\in\Gamma} f_e \prod_{\ell\in \Gamma} f_\ell\right)\right]_{T^d}$$

The factors $f_v, f_e, f_\ell$ will be power series in $\psi$ and $\kappa$ classes local to the given data.  These factors will all involve the $A$ and $B$ series.  

On the one hand, anything involving a sum over all possible dual graphs will be complicated.  But, in some other sense, this is the simplest such method of doing this that one might hope for.
For a graph $\Gamma$, let $\{\zeta_v|v\in V(\Gamma)\}$ be formal variables satisfying $\zeta_V^2=1$.  The $f_v, f_e, f_\ell$ will involve some $\zeta_v$ terms.  We then multiply our whole product at the end by $\prod\zeta_v^{g_v+1}$.  

The formula will also become nicer if we divide by $2^{b_1(\Gamma)}$.
Vertex Factor:
$$f_v=\exp(-\{\log(A(\zeta_vT))\})$$
Leg Factor:
$$f_\ell=B(\zeta_v\psi_\ell)$$
Edge Factor
$$f_e=\frac{A(\zeta_1\psi_1T)\zeta_2 B(\zeta_2\psi_2T)+\zeta_1B(\zeta_1\psi_1T)A(\psi_2\zeta_2T)+\zeta_1+\zeta_2}{\psi_1+\psi_2}$$

\begin{theorem}[Pandharipande-Pixton-Zvonkine, Janda]
$$R(g,n,d)=0$$
\end{theorem}

\begin{conjecture}
$\{R(g,n,d)\}$ together with pushforward and pullback by gluing and forgetful morphisms give all tautological relations.
\end{conjecture}

Motivation: this gives all known relations.

In response to question from Renzo, outlines how you get FZ relations from these -- parts where your partition are 1 mod 3 correspond to having a marked point, with some power of psi, and pushing away that point, parts of the partition that are divisible by 3 are slightly more complicated.  Setting $n=0$ gives you the FZ relation corresponding to the empty partition.

\end{document}
