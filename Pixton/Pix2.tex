\documentclass{amsart}

\theoremstyle{definition}
\newtheorem{theorem}{Theorem}
\newtheorem{definition}{Definition}
\newtheorem{lemma}{Lemma}
\newtheorem{example}{Example}
\newtheorem{question}{Question}
\newtheorem{conjecture}{Conjecture}
\newtheorem{corollary}{Corollary}

\newcommand{\C}{\mathbb{C}}
\newcommand{\one}{\text{one}}
\newcommand{\End}{\text{End}}
\newcommand{\Mbar}{\overline{\mathcal{M}}}
\newcommand{\M}{\mathcal{M}}
\newcommand{\Q}{\mathbb{Q}}

\newcommand{\Aut}{Aut}


\title{The tautological ring and cohomological field theories II}
\author{Aaron Pixton}
\begin{document}
\maketitle

Today, we develop some of the classification theory of Cohomological field theories.

\section{Cohomological field theories}

\begin{definition}
A \emph{cohomological field theory} begins with the data of 
\begin{itemize}
\item $V$ a finite dimensional vector space over $\C$.
\item $\eta$ a nondegenerate symmetric bilinear form on $V$. 
\end{itemize}
and gives a family
$$\Omega_{g,n}\in H^*(\Mbar_{g,n})\otimes (V^*)^{\otimes n}$$
satisfying
\begin{enumerate}
\item $\Omega_{g,n}$ is $S_n$ invariant
\item Formulas for pullbacks of $\Omega_{g,n}$ under gluing maps
\end{enumerate}

If furthermore, there exists a $\one\in V$ such that 
$$\tau^*(\Omega_{g,n}(v_1\otimes\cdots\otimes v_n)=\Omega_{g,n+1}(v_1\otimes\cdots\otimes v_n\otimes\one)$$
and
$$\eta(v_1,v_2)=\Omega_{0,3}(v_1\otimes v_2\otimes\one)$$
then we say $\Omega$ is a \emph{CoHFT with unit}.
\end{definition}

We are mostly interested in CohFT with unit because

\begin{definition}
The quantum product $\bullet$ on $V$ is given by

$$\Omega_{0,3}(v_1\otimes v_2\otimes v_3)=\eta(v_1\bullet v_2,v_3)$$
\end{definition}

If $\Omega$ is a CohFT with unit then $(V,\bullet)$ is a commutative $\C$-algebra.

\begin{definition}
We say $\Omega$ is \emph{semisimple} if $(V,\bullet)$ is semisimple, i.e. $V\cong \bigoplus\C\epsilon_i$, with $\epsilon_i\epsilon_j=\delta_{ij}\epsilon_i$
\end{definition}

The goal today is to explain a reconstruction theorem that reproduces a semisimple CohFT from a very small amount of data; this will give a classification of semisimple CohFT.  Later, we will apply this to the tautological ring.

\section{Classification of semisimple CohFTs}

The main idea of the classification will be to construct a group action on the set of all semisimple CohFTs; this will be Givental's R-matrix action.

Let $R(z)\in\End(V)[[z]]$, i.e. $R(Z)=1+R_1z+R_2z^2+\dots$

We require $R(z)$ to satisfy the \emph{symplectic condition}:
$$
v\eta(R(z)v, R(-z)w)=\eta(v,w)$$
The set of all $R$ satisfying these conditions form a group, sometimes called the symplectic loop group.  Our maing goal define an action of this group on the space of semisimple CohFTs.


\subsection{Warm-up group action}
We begin by defining a different action on the space of all CohFTs that does not preserve the semisimple ones.

Let $\Omega$ be a CohFT on $(V,\eta)$.  We define a new CohFT $R\Omega$ by

$$(R\Omega)_{g,n}(v_1\otimes\cdots\otimes v_n)
=\sum_{\Gamma \text{stable graphs for} \Mbar_{g,n}} \frac{1}{|\Aut(\Gamma)|}
\iota_{\Gamma*}(\prod_{v\in\Gamma} \Omega_{g_v, n_v}( S_v)$$
Where $S_v$ is defined by:
\begin{itemize}
\item At the $i$th leg $\ell$, we place $R^{-1}(\psi_\ell)(v_i)$.
\item At each edge, we place $$\Delta=\frac{\eta^{-1}-R^{-1}(\psi_1)\eta^{-1}R^{-1}(\psi_2)^t}{\psi_1+\psi_2}$$
\end{itemize}

The fact that the numerator of $\Delta$ is divisible by the denominator is equivalent to the symplectic condition.

This should look similar to what was happening at the end of the first lecture.

DRAWING FOR INTUITION

Another observation is that $R$ does not appear in this expression, but $R^{-1}$; Givental originally used $R$, we use $R^{-1}$ so that we get a left group action instead of a right group action.

To see that this is a CohFT, there are essentially two things that need to be checked.

First, it is clear that this result is $S_n$ invariant.

Second, we must check that when we pull back to a boundary divisor, the appropriate identity holds.

Here's how that works -- suppose first we pull back to a tree divisor.

There are essentially two things that can happen.

One is that we pick a separating edge and cut across that.

When this happens, we multiply by $-(\psi_1+\psi_2)$, and so get a contribution of $-(\psi_1+\psi_2)\Delta=-(\eta^{-1}+R^{-1}(\psi_1)\eta^{-1}R^{-1}(\psi_2)^t$

Another contribution is when we split a vertex in half; here, we can use the fact that we know $\Omega$ is a CohFT, and hence we know how $\Omega$ splits.  When we do this, we insert $\eta^{-1}$.  

This $\eta^{-1}$ contribution cancels with that from the first type of contribution, and we are left with simply $R^{-1}(\psi_1)\eta^{-1}R^{-1}(\psi_2)^t$, which is essentially what we get from following the definitions.

We must also check that this gives a group action, which is just careful bookkeeping of what happens when we apply two of these in a row.

This action does not preserve having a unit even, much less semisimplicity.

\subsection{Another warm-up group action}

We now describe another action that does not preserve having a unit, the translation action.

Let $T(z)$ be a $V$-valued power series, with

$$T(z)=T_2z^2+T_3z^3+\cdots$$
i.e., $T$ has no constant or linear terms.

For $\Omega$ a CohFT, define a CohFT $T\Omega$ by

$$(T\omega)_{g,n}(v_1\otimes\cdots\otimes v_n)=\sum_{m\geq 0} \frac{1}{m!} (p_m)_*\Omega_{g,n+m}\left(v_1\otimes\cdots v_n\otimes T(\psi_{n+1})\otimes\cdots\otimes T(\psi_{n+m})\right)$$
where $p_m:\Mbar_{g,n+m}\to \Mbar_{g,n}$ is the forgetful map.

This appears to be an infinite sum, but is only a finite sum because $T$ starts in degree two, because increasing $m$ increases the cohomological degree by 2 but only increases the dimension by 1.

Graph theoretically, this is much easier -- we have a single vertex, and are adding some number of new legs that we will forget about, on which we add $T$.

Essentially what we are doing is adding in $\kappa$ classes here.

Again, we must check that $T\Omega$ is a CohFT and that this is a group action.

To keep track that we are a CohFT, we restrict to a boundary divisor, which splits our first $n$ marking in some ways.  We have choices of how we're going to add the next $m$ points.  These give multinomial coefficients of how we choose to divide them, and the $\frac{1}{m!}$ terms exactly account for that.

\subsection{Combining our actions}

We have $\End(V)$ valued action and an $V$ valued action; we combine them into an action of affine transformations of $V$.

Checking this, amounts to checking that

$$RT\Omega=(RT)(R\Omega)$$

\begin{theorem}
If $\Omega$ is a CohFT with unit $\one$ and $R(z)$ satisfies the symplectic condition, then $RT\Omega$ is also a CohFT with unit $\one$ where
$$T(z)=z(\one-R^{-1}(z)\one)$$
\end{theorem}
\begin{proof}
Lengthy but simple bookkeeping, using some formulas about how to pull back $\psi$ classes.
\end{proof}

\begin{definition}
The unit preserving $R$-matrix action is 
$$R_\bullet(\Omega)=RT\Omega$$
for $T$ as above.
\end{definition}

What does $R_\bullet$ look like.  First we're translating, which is really just adding these $\kappa$ classes.  
Then the $R$ matrix action is twist what is happening at every leg and vertex by $R$.

There are really three things happening, at vertex, edge, and legs.

These three factors are going to correspond to the three types of local factors appearing in the previous talk.

There is still the question of what this action is good for.
\section{Reconstruction theorem}

\begin{theorem}[Givental, Teleman]
The (unit-preserving) $R$-matrix action is regular (i.e., free and transitive) on the set of CohFTs with $(V,\bullet)$ semisimple and certain data about them fixed.  Specifically, we fix $(V,\eta,\one,\bullet)$.
\end{theorem}

\begin{proof}
First, let's check it preserves all the data.

We've already seen it fixes the unit.

To fix $\bullet$, we're really only looking at cohomological degree 0 pieces, and the $R$-matrix action is easy to see that $R$ preserves the degree 0 pieces.
\end{proof}


As stated, the reconstruction theorem is not very explicit, as it doesn't tell us how to find the $R$-matrix.  There is a more explicit one, that, given certain hypotheses, gives a differential equation that determines the $R$ matrix in terms of some small amount of data.

Even with that, there is a question of what it means to determine the $R$-matrix.

\begin{definition}
A CohFT $\omega$ is called a \emph{topological field theory} if it is supported in cohomological degree 0
\end{definition}

Note that any CohFT has an underlying topological field theory.  
\begin{lemma}
The data $(V,\eta,\one,\bullet)$ uniquely determines a topological field theory.
\end{lemma}
\begin{proof} 
Note that $\omega_{0,3}(v_1\otimes v_2\otimes v_3)=\eta(v_1\bullet v_2,v_3)$ are given.

Everything else can be determined by this and the fact that we're cohomological degree o.

It will be useful to be very explicit about this in the semisimple case.

Assuming semisimplicity, we have $V=\bigoplus \C \epsilon_i$, with $\epsilon_i$ orthogonal idempotents.

The only extra data we have is $X_i=\eta(\epsilon_i,\epsilon_i)$.

Then 

$$\omega_{g,n}(\epsilon_{i_1}\otimes\cdots\otimes\epsilon_{i_n})=\left\{\begin{array}{ll} \sum_{i=1}^N X_i^{1-g} & n=0 \\
x_i^{1-g} & n>0, i_1=i_2=\cdots=i_n \\
0 & \text{else}
\end{array}
\right.
$$ 

Thinking about how this works, is a boundary strata that is zero dimensional is giving a three regular graph, with genus 0 everywhere.

The rule for how we restrict, is inserting the bivector $\eta^{-1}$ at each edge.  In the idempotent basis, this will give us an $X^{-1}$.  Each vertex we need to have the same three $\epsilon_i$ appearing, and so will contribute $x_i$.

\end{proof}

So, given any CohFT, there is one very nice CohFT (namely, the topological one described) that has the same $(V, \eta, \one,\bullet)$.  Thus, to describe a general semisimple CohFT, we can describe it by describing the $R$-matrix that takes it to our nice one.

\begin{corollary}
A semisimpe CohFT is tautological (in cohomology)
\end{corollary}

\begin{proof}
The topological CohFT is tautological, and the $R$-matrix action preserves being tautological.
\end{proof}

We don't know if the same thing is true in Chow.

One piece still owed is the addendum of how to make finding the $R$-matrix explicit; we put that on hold for now.

\section{Tautolgoical relations}

How does this machinery lead to tautological relations?

Witten's $r$-spin class gives a CohFT $W_{g,n}(v_{a_1},\dots, v_{a_n})\in H^*(\Mbar_{g,n})$.

The state space $V=\langle v_0,v_1,\dots, v_{r-2}\rangle$.
 
If we feed in vectors in this special set of generators then $W_{g,n}(v_{a_1},\dots, v_{a_n})\in H^d(\Mbar_{g,n})$ for $d=((r-2)(g-1)+\sum a_i)/r$, is of \emph{pure codimension}.

IDEA: apply the reconstruction theorem to get an explicit formula for $W_{g,n}$ and then all components of dimension of cohomological degree different than $d$ gives a tautological relation.

PROBLEM: $W_{g,n}$ is not semisimple (at the origin)

SOLUTION: Look at a shifted version of $W_{g,n}$ instead.  The shifted version will no longer be of pure degree $d$, but will have at most that cohomological degree; i.e., it will live in $H^{\leq d}(\Mbar_{g,n})$.  We can choose a shifted version that is semisimple, and then get tautological relations in degree greater than $d$.

Note that Witten's $r$-spin action makes sense for any $r\geq 2$.  For $r=2$, there does not exist a shift that is semisimple.

So $r=3$ is the first interesting case, here the degree looks like $g/3+\text{stuff}$, and so looks similar to the Faber-Zagier relations, and is the motivation for using $3$-spin curves.

The shift we make is very similar to the translation action.

Let $\gamma\in V$.

We define the shifted theory by

$$W^\gamma_{g,n}(v_1,\dots, v_n)=\sum_{m\geq 0}\frac{1}{m!}(p_m)_*\left(W_{g,n}(v_1,\dots, v_n,\gamma,\dots, \gamma)\right)$$

On the one hand, this is simpler than the previous shift because it does not involve $\psi$ classes.  It should be thought of simply shifting the origin of the Frobenius manifold.

On the other hand, we have to be careful about the infinite sum.

However, since we know the precise degree of $W_{g,n}$ we can check again that this will be a finite sum, because inserting each $\gamma$ decreases the cohomological degree by at least 1.

Note we have made some choices; we chose $r$; we could choose another CohFT with similar homogeneity conditions.

We also chose $\gamma$, that we need to be semisimple.  It turns out for $r=3$ there is a unique shift that works, that give us exactly the tautological relations we got before.  For $r>3$, the choice is no longer unique, and so we get other families of tautological relations that are somehow related.

\end{document}
