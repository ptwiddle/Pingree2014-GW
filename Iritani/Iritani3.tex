\documentclass{amsart}
\usepackage{tikz}
\linespread{1.2}

\theoremstyle{definition}
\newtheorem{dummy}{}[section]
\newtheorem{remark}[dummy]{Remark}
\newtheorem{theorem}[dummy]{Theorem}
\newtheorem{definition}[dummy]{Definition}
\newtheorem{example}[dummy]{Example}
\newtheorem{question}[dummy]{Question}
\newtheorem{conjecture}[dummy]{Conjecture}
\newtheorem{lemma}[dummy]{Lemma}
\newtheorem{proposition}[dummy]{Proposition}
\newtheorem{corollary}[dummy]{Corollary}
\newcommand{\TT}{\mathbb{T}}
\newcommand{\Pic}{\text{Pic}}
\newcommand{\Z}{\mathbb{Z}}
\newcommand{\R}{\mathbb{R}}
\newcommand{\GIT}{//}
\newcommand{\X}{\mathcal{X}}
\newcommand{\Stab}{\text{Stab}}
\newcommand{\Spec}{\text{Spec}}
\newcommand{\aff}{\text{aff}}
\newcommand{\I}{\mathcal{I}}
\newcommand{\OO}{\mathcal{O}}
\newcommand{\Map}{\textrm{Map}}
\newcommand{\proj}{\mathbb{P}}
\newcommand{\J}{\mathcal{J}}
\newcommand{\sm}{\text{sm}}
\newcommand{\Mbar}{\overline{\mathcal{M}}}
\newcommand{\M}{\mathcal{M}}
\newcommand{\Q}{\mathbb{Q}}
\newcommand{\C}{\mathbb{C}}
\newcommand{\Age}{\text{Age}}
\newcommand{\Fix}{\text{Fix}}
\newcommand{\one}{1}
\newcommand{\st}{\text{st}}
\newcommand{\ttt}{\mathbb{t}}
\newcommand{\ev}{\text{ev}}
\newcommand{\vir}{\text{vir}}
\newcommand{\Quot}{\text{Quot}}
\newcommand{\Hom}{\text{Hom}}


\newcommand{\Aut}{Aut}




\author{Hiroshi Iritani}
\title{Toric Mirror Symmetry III}


\begin{document}

\maketitle

Last time we discussed the relation between quantum cohomology, the quantum differential equation, and the lagrangian cone, which was all very abstract.

Today we will move to mirror symmetry on toric stacks, which should hopefully be more concrete.


\section{B-model}


\subsection{Toric stacks}
Consider a toric stack $X$; we assume $X$ to be compact.  Borisov-Chen-Smith described how the commbinatorial data of a stacky fan encodes $X$.  

Begin with
\begin{itemize}
\item  $N$ a finitely generated abelian group
\item $\Sigma$ a simplicial fan in $N\otimes\R$
\item $\beta:\Z^{m^\prime}\to N$, so that $\beta(e_1),\dots,\beta(e_{m^\prime})$ are distinct generators that generate all one dimensional cones; they need not be primitive.
\end{itemize}

We will assume that following:
\begin{itemize}
\item $N=\Z^n$ has no torsion; this is only to make the descriptions less technical
\item The support of $\Sigma=N\otimes\R$; so $X$ is compact
\item $\sigma$ admits a strictly convex piecewise linear function:
$$\eta:N\otimes\R\to\R$$
this gives an ample line bunle, making $X$ projective
\item Let 
$$\Delta=\text{Conv}\left(\beta(e_1),\dots,\beta(e_{m^\prime})\right)$$
 be the fan polytope, we assume that $\beta(e_i)\in\partial\Delta$; this means that $c_1(X)\geq 0$.  Mirror symmetry can be stated without this condition, but it becomes more complicated.
\end{itemize}
\subsection{The mirror of $X$}

Let 
$$\Delta\cap N=\{b_1,\dots,b_m\}$$
So $m\geq m^\prime$ the number of generators.

We define
$$f_c(x)=\sum_{i=1}^m c_ix^{\vec{b}_i}$$
with
$$x^{\vec{b}_1}=x_1^{b_{i_1}}\cdots x_n^{b_{i_n}}$$
Let
$$\tilde{\M}=\left\{(c_1,\dots, c_m)|c_i\neq 0 \text{ if $b_i$ is a vertex}\right\}$$
The condition here guarantees that the newton polytope of $c$ is always $\Delta$.

We have a torus $(\C^*)^n$ acting on $\tilde{\M}$ by scaling the $x$ variables; the resulting quotient $\M=\tilde{\M}/(\C^*)^n$ will be our mirror manifold.

\subsection{Oscillating integrals}
The mirror will be

$$\int_{\substack{\text{noncompact}\\ \text{cycle}}} e^{f_c(x)/z}\frac{dx_1}{x_1}\dots\frac{dx_n}{x_n}$$

this will correspond to the $\J$ function of quantum cohomology

\begin{remark}
If $X_1, X_2$ have the same fan polytope $\Delta_1=\Delta_2$, we see that $X_1$ and $X_2$ have the same mirror.  

Geometrically, having the same fan polytope implies that $X_1$ and $X_2$ are related by a crepant tranformation.
\end{remark}

\subsection{Gauss-Manin system of $f_c(x)$}
We will give two different descriptions of the Gauss-Manin section.

A topological description will give it a $\Z$ structure and make it easy to describe the paring.

An algebraic description will make it easy to see the differential equations (which in this case will be a GKZ system), and also (this is a little technical) describe the extension to $z=0$ explicitly.

\subsubsection{Topological description}


Here we don't just consider the fibers of $f_c$, we consider some relative (co)homology groups.  

Define.
$$R^\vee_{c,z}=H_n\Big((\C^*)^n,\big\{x\big|\text{Re}(f_c(x)/z)\leq -M\big\}\Big)$$
For $M>>0$, this group will not depend on $M$.

For generic $c$, this spaces are nice, but for some few bad values of $c$ the homology will not be nice.

\begin{definition}
We say $f_c(x)$ is \emph{nondegenerate} if for any face $F$ of $\Delta, (0\leq \dim F < b)$, then
$$f_{F,c}(x)=\sum_{b_i\in F}c_i x^{\vec{b}_i}$$
has no critical points on $(\C^*)^n$. 
\end{definition}

DRAWING OF POLYTOPE FROM THE F condition -- coefficient 1 on each vertex, coefficient 0 on the $b_1$ in the middle of the edge.

If we restrict to the face with the extra point on it, we get the polynomial
$$x+cy+\frac{y^2}{x}$$ which we see has no critical points as long as $c\neq \pm 2$, in which case we have a square.

In general, if we have the $A_n$ resolution, then we get a similar description; with there not being critical points unless the $c_i$ are the correct binomial coefficients.

We may view the polynomials $f_{F,c}(x)$ as being the polynomials at infinity on the boundary, and so we are saying that there are no critical values at infinity.

\begin{theorem}[Kouchnirenko]
\begin{enumerate}
\item $\{c|f_c \text{ is nondegenerate}\}$ is a non-empty open set
\item If $f_c$ is nondegenerate, then the critical points of $f_c$ are isolated and $\dim \text{Jac}(f_c)=\text{vol}(\Delta)$, where we have normalized so that the volume of the standard $n$-simplex is equal to one.
\end{enumerate}

\end{theorem}
Here, $\text{Jac}(f_c)$ is the Jacobian ring; its dimension is the number of critical points counted with multiplicity.

\begin{example}
The ongoing example of the resolution of $\proj(1,1,2)$ has volume 4, as each of the four top dimensional cones in the fan is a standard simplex.
\end{example}


\subsubsection{Morse theory for non-degenerate $f_c$}

Using Morse theory on the non-degenerate values, we see that our homology groups will be generated by Lefschetz thimbles:
$$R^\vee_{c,z}\cong\bigoplus_{i=1}^N \C \Gamma_i$$

BIG PICTURE:


At bottom, have $\C$-plane, the image of $f_c(x)/z$; above it, we have $(\C^*)^n$.  We have the critical values in the $\C$-plane marked; above them, we have just a simple cone singularity.  Taking a a line from the critical values to $\text{Re}(z)<<0$, we can take deform the critical point away along this line; it will be an $S^{n-1}$ over most points (the vanishing cycle), and so the give us a Lefschetz thimble $G_i\cong \R^n$.

Straight from the definition of $R^\vee_{c,z}$ as a homology group, we get a Gauss-Manin connection, and so $\cup_{c,z} R^\vee_{(c,z)}$ form a local system over
$\tilde{M}^0\times\C^*\subset \tilde{M}\times\C^*$


\subsubsection{Pairing}

To describe the pairing, we construct the dual basis by, in our picture, taking the lines from the critial points to the direction where $\Re(z)$ is very large.

In other words, rather than:
$$H_n\Big((\C^*)^n,\big\{x\big|\text{Re}(f_c(x)/z)<<0\big\}\Big)$$
we work with
$$H_n\Big((\C^*)^n,\big\{x\big|\text{Re}(f_c(x)/z)>>0 \big\}\Big)$$
The dual Lefschetz thimble $\Gamma^\vee_i$ intersects $\Gamma_i$ exactly at the critical point.

The pairing 
$$I:R^\vee_{c,-z}\otimes R^\vee_{c,z}\to\C$$
and similarly, without the  $\vee$.  

Then it is immediate from the definition and the above discussion that
\begin{itemize}
\item $I$ is perfect over $\Z$
\item $I$ is flat
\end{itemize}

\subsection{Algebraic description}
This description is due to Dwonk, K. Saito, Sabbah,

We consider
$$\phi(x)e^{f_c(x)/z}\frac{dx}{x}$$
for $\phi\in\C[x_1^\pm,\dots, x_n^\pm]$
which gives a section of $R\otimes_\C\OO$ via integration over $\Gamma_i$.

Note that $\Gamma_i$ is noncompact, but the exponential factor means the integrand decays exponentially as we tend toward infinity.

We want to work modulo exact forms.

\subsubsection{Exact forms}
An elementary calculation gives:
$$d\left[g(x)e^{f_c(x)/z}\frac{dx_1}{x_1}\cdots\hat{\frac{dx_i}{x_i}}\cdots\frac{dx_n}{x_n}\right]
=\left(x_i\frac{\partial g}{\partial x_i}+\frac{1}{z} g(x) x_i\frac{\partial f_c}{\partial x_i}\right)e^{f_c/z}\frac{dx}{x}
$$

We define $$R^{(0)}=\OO_{\tilde{\M}}[z][x_1^\pm,\dots,x_n^\pm]\Big/ \left\langle 
zx_i\frac{\partial g}{\partial x_i}+g x_i\frac{\partial f_c}{\partial x_i} \Big| g\in\OO_{\tilde{\M}}[z][x^\pm], 1\leq i \leq n\right\rangle$$

The zero here means that we have done the extension to $z=0$; topologically, this was not well defined, but algebraically it makes sense.


At $z=0$, we have
$$R^{(0)}\big|_{z=0}=\cup_c \text{Jac}(f_c)$$
and, by the description of this as integrals over Lefschetz thimbles, we see that

$$R^{(0)}\big|_{\tilde{\M}^0\times\C^*}\cong R\quad\leftarrow\quad\text{topological}$$

Also, the Gauss-Manin connection has formula

\begin{align*}
\nabla_{\frac{\partial}{\partial c_i}}\left[\phi e^{f_c/z}\frac{dx}{x}\right]&=
\left[\left(\frac{\partial\phi}{\partial c_i}+\frac{1}{z}x^{b_i}\phi\right)e^{f_c/z}\frac{dx}{x}\right] \\
\nabla_{z\frac{\partial}{\partial z}}\left[\phi e^{f_c/z}\frac{dx}{x}\right]&=
\left[\left(z\frac{\partial\phi}{\partial z}+\frac{1}{z}\phi\right)e^{f_c/z}\frac{dx}{x}\right] 
\end{align*}

\subsubsection{GKZ system}
This gives the better behaved GKZ system of Borisov-Horja.

Let

$$\varphi_e=\left[x^e e^{f_c/z}\frac{dx}{x}\right]$$
for $e=(e_1,\dots, e_n\in\Z^n$.
We see
\begin{align*}
z\nabla_{\frac{\partial}{\partial c_i}}\varphi_e&=\varphi_{e+b_i} \\
\sum_{i=1}^m b_{ij}\nabla_{c_i\frac{\partial}{\partial c_i}}\varphi_e&=e_j\varphi_{e}
\end{align*}
which imply that the Gauss-Manin connection descends to $\tilde{\M}/(\C^*)^n$.


\begin{itemize}
\item $R^{(0)}\Big|_{\tilde{\M}^0\times\C}$ is locally free of rank $\text{vol}(\Delta)=\dim H_{CR}^*(X)$
\item The pairing on $R$ extends to $z=0$:
$$( - )^* R^{(0)}\otimes R^{(0)}\to\OO$$
with
\begin{align*}
(-):\tilde{\M}^0\times\C &\to \tilde{\M}^0\times\C \\
(c,z)&\mapsto (c,-z)
\end{align*}
\end{itemize}

The pairing is defined by

$$I\left(\left[\phi_1e^{-f_c/z}\frac{dx}{x}\right], \left[\phi_2e^{f_c/z}\frac{dx}{x}\right]\right)
=\sum_{i=1}^N\left(\int_{\Gamma_i^\vee} \phi_1e^{-f_c/z}\frac{dx}{x}\right)\left(\int_{\Gamma_i^\vee} \phi_2e^{f_c/z}\frac{dx}{x}\right)$$
and, taking the stationary phase asymptotic, (i.e., taking $z\to 0$ with $\text{arg}(z)$ fixed we have that this is equal to
$$
=\sum_{i=1}^N (2\pi i z)^n (-1)^{\frac{n(n-1)}{2}}\left(\frac{\phi_1(cr_i)\phi_2(cr_i)}{\text{Hess}(f_c)(cr_i)}+O(z)\right)$$
where the $cr_i$ are the $i$ critical points.

Here $$\frac{\phi_1(cr_i)\phi_2(cr_i)}{\text{Hess}(f_c)(cr_i)}$$ is the residue pairing in the Jacobi ring.

This follows from

$$\int_{\Gamma_i} \phi e^{f_c(x)/z}\frac{dx}{x}\sim (-2\pi z)^{n/2} e^{f_c(cr_i)/z} \frac{\phi(cr_i)}{\sqrt{\text{Hess}(f_c)(cr_i)}}+\cdots$$
and playing around with signs and whatnot.  

Define $Q_B=\frac{1}{(2\pi i z)^n} I$ extends to $z=0$ and is non-dgenerate there.

We define
$$\nabla=\nabla^{GM}-\frac{n}{2}\frac{dz}{z}$$ 
so that that the modified pairing $Q_B$ will be flat with this modified connection.

These modifications have something to do with $\mu=\frac{1}{2}\text{deg}-\frac{n}{2}$


\section{Summary}
\begin{itemize}
\item $R^{(0)}$ locally free on $\M^0\times\C$.
\item A flat connection with a pole along $z=0$
$$\nabla:R^{(0)}\to R^{(0)}(\M^0\times\{0\}) \otimes \left(\pi^*\Omega^1_{\M^0}\oplus\OO_{\M^0\times\C}\frac{dz}{z}\right)$$
here $\pi$ is the projection $\pi:\M^v\times\C\to\M^0$.
\item $Q_B:(-)^*R^{(0)}\otimes R^{(0)}\to\OO$
\item $R_\Z\subset R^{\nabla}$ relative cohomology with $\Z$ coefficients shifted by $z^{-n/2}$.
\end{itemize}
We might call this 4-tuple a TE$\Z$P, structure, after Hertling's TERP structure, where $R$ was real, and $P$ was pairing, and TE was twistor something?

There is also a noncommutative hodge structure (Katzarkov-Kontsevich-Pantev)

\subsection{Mirror symmetry isomorphism}

Take $X$ a toric stack as above; there exist a ``large radius neighborhood'' $U_X\subset\M^0$, and a mirror map
$$\tau_X:U_X\to H^{\leq2}_{CR}(X)$$
(technically, we need to take the universal cover of $U_x$). 

The semi-positive condition ensures that everything is analytic, otherwise we must work formally.

We have $(F,\nabla, Q_A)$ the quantum connection of $X$ ($Q_a$ is the Poincare pairing).

\begin{theorem}[Givental, Coates-Corti-Iritani-Tseng, Cheong-Ciocan-Fontanine-Kim, Iritani, Reichel-Sovenheck]
\begin{align*}
\tau_X^*(F,\nabla,Q_A)&\cong (R^{(0)},\nabla,Q_B)\big|_{U_X\times\C}\\
\one &\mapsto\varphi_0=\left[e^{f_c(x)/z}\frac{dx}{x}\right]
\end{align*}
\end{theorem}

\begin{corollary}
If $X_1, x_@$ have the same $\Delta$, then the Quantum connection of $X_1$ is isomorphic to that of $X_2$ analytically continued along $\M^0$.

Another way of saying this is that there exists a symplectic transformation 
$$\mathbb{U}:\mathcal{H}_1\stackrel{\cong}\to\mathcal{H}_2$$
with
v$$\mathbb{U}(\mathcal{L}^{an}_1)=\mathcal{L}_2^{an}$$


\end{corollary}

We will explain how the lagrangian cone is the same as the quantum connection, and can be viewed as the image of the fundamental solution under the quantum connection.


We have $(\pi_* R^{(0)}, \nabla)$ contained in $\mathcal{H}_1$ over $U_i$ via $L_i^{-1}$, and these inclusions form a commutative diagram with $\mathbb{U}$, which is analytic continuation of flat sections.

\subsection{Proofs}
An important ingredient is:

\begin{theorem}[Mirror Theroem]
$$\I(y)=\J(\tau(y))$$
where $\I$ is an explicit cohomology-valued function on $U_x\subset \M^0$.
\end{theorem}

The mirror theorem was proven a long time ago by Givental, and Cheong-Cican-Fontaninte-Kim generalized to toric stacks; then the
bbGKZ maps both to $(R^{(0)},\nabla)$ and $\tau^*(F,\nabla)$.  The I function $\I$ is a complete set of solutions to the GKZ system, and we show that each of these maps are isomorphisms.

The matching of the pairing $Q_A=Q_B$ is a more subtle problem, but essentially there is some uniqueness of the pairing compatible with the monodromy around $q_i$ axis.  $U_X$ is the complement of normal crossing divisors, and the quantum connection has logarithmic singularities on these divisors.  It turns out there is a unique metric, up to a constant, around each of these points that is compatible with the corresponding monodromy, and so you only need to check one value of the pairings match.

Another way to prove this is to use some equivariant versions of these, and then use stationary phase asymptotics on the $B$-model side will match with the localization calculation

\subsubsection{}

COMMUTATIVE DIAGRAM: $\Theta(y,z)$ isomorphism betwee $(F_i, \nabla_i, Q_i)$, which each map to $\mathcal{H}_i$ via $L_i^{-1}$, and $\mathbb{U}(z)$ is an isomorphism between the $\mathcal{H}_i$.

So,
$$L_2^{-1}(y,z)\Theta(y,z)=\mathbb{U}(z)L_1^{-1}(y,z)$$
note that $\Theta$ depends on $y$ by $\mathbb{U}$ does not.

The left hand side can be viewed as a Birkhoff factorization of the right hand side because $L_2^{-1}$ is regular on $z\in\proj^1\setminus\{0\}$, and $\Theta$ is regular for $z\in \C$.

In general, you can show that $\mathbb{U}(z)$ does not contain positive powers of $z$ then $\Theta$ is constant, and this shows the isomorphism of Frobenius manifolds.  


\end{document}
